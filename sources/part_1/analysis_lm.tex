
\chapter{On the Scaling Laws of Geographical Representation in Language Models}
\label{chap:geobias}


\subsection{Introduction \& Related work}

In recent years, numerous studies analyzing the hidden representations of self-supervised language models have provided insights into how these models incorporate linguistic knowledge from their training data \citep{gupta-etal-2015-distributional,kohn-2015-whats,shi-etal-2016-string,ijcai2018p796,conneau-etal-2018-cram,jawahar-etal-2019-bert}. 

This line of work has been called probing, as most approaches are generally based on the training of classifiers---or \textit{probes}---upon frozen hidden representations.

Analyzing the representations of language models can point out sociocultural biases that were inherently learned by the models during training \citep{zhao-etal-2018-gender}, and training probes can help with mitigating these biases \citep{ravfogel-etal-2020-null, iskander-etal-2023-shielded}.

Among probing tasks, several works have focused on geographical representations that are implicitly embedded in language models. \citet{lotr} show that coordinates of places in the Middle-Earth can be predicted by just using the co-occurence matrix extracted from the Lord of the Rings novels. \citet{faisal-anastasopoulos-2022-geographic} build networks from geographical representations based on monolingual and multilingual models of different sizes. They show that all models embed more accurate geographical representations for countries of the Global North.

This geographical discrepancy can be explained by biases that are inherent to the datasets used for pretraining \citet{faisal-etal-2022-dataset}. Imbalanced frequency distributions of geographical references in pretraining data causes distortions in the representational space \citep{zhou2021freqbased}. These distortions lead to a loss in the models' ability to differentiate between under-represented locations.

Recently, \citet{gurnee2023language} have probed large language models from the Llama-2 suite \citep{touvron2023llama} to extract coordinates of prompted locations from hidden representations across layers. They show that models ranging from 7B to 70B parameters are able to convincingly embed geographical coordinates on a world map when representing basic prompts.

In this work, we propose to extend the analysis by \citet{gurnee2023language} to smaller language models, in order to observe how scale affects the ability of models to implicitly embed geographical information from raw training data. We show that such ability consistently improves with model size, and that even tiny models are able to produce visually meaningful world maps.

We make several contributions:
\begin{itemize}
    \item We show that geographical information can be extracted to a certain extent from representations at every model scale;
    \item We observe that larger models are more geographically biased than their smaller counterparts;
    \item We find that the performance of models in terms of geographical probing is correlated with the frequency of corresponding country names in the training data.
\end{itemize} 

\subsection{Scaling Laws of Geographical Probing}
\label{sec:scaling}
\begin{figure*}[h]
    \centering
    \begin{subfigure}[b]{0.43\textwidth}
         \includegraphics[trim={0 0 0 0.7cm},clip,width=\linewidth]{sources/part_1/geographical/imgs/pythia-14m.png}
         \caption{Pythia 14M ($R^2 = 34.34$)}
         \label{fig:14m_map}
         \vspace{1em}
    \end{subfigure}
    \begin{subfigure}[b]{0.43\textwidth}
         \includegraphics[trim={0 0 0 0.7cm},clip,width=\linewidth]{sources/part_1/geographical/imgs/pythia-160m.png}
         \caption{Pythia 160M ($R^2 = 55.28$)}
         \label{fig:160m_map}
        \vspace{1em}
    \end{subfigure}
    \begin{subfigure}[b]{0.43\textwidth}
         \includegraphics[trim={0 0 0 0.7cm},clip,width=\linewidth]{sources/part_1/geographical/imgs/pythia-1b.png}
         \caption{Pythia 1B ($R^2 = 67.94$)}
         \label{fig:1b_map}
    \end{subfigure}
    \begin{subfigure}[b]{0.43\textwidth}
         \includegraphics[trim={0 0 0 0.7cm},clip,width=\linewidth]{sources/part_1/geographical/imgs/pythia-2.8b.png}
         \caption{Pythia 2.8B ($R^2 = 74.97$)}
         \label{fig:2.8b_map}
    \end{subfigure}
    \caption{Predicted coordinates of test set instances for different model sizes. Each color represents a different continent.}
    \label{fig:maps}
\end{figure*}

In this section, we train geographical probes for a wide variety of models at different scales.

\subsubsection{Methodology}

We use the World dataset from \citet{gurnee2023language} as a geographical data source. It contains 39,504 location names from the whole world along with corresponding longitude and latitude. We use the same train-test split strategy as in the original article, thus keeping 20\% of samples for testing purposes.

For each location name $X$, we prompt models with the text: ``\textit{Where is $X$ in the world?}''. We then infer with a given model on the whole dataset, and use the last token belonging to the entity $X$ as the model's representation. To follow the linear probing paradigm used in \citet{gurnee2023language}, we train a Ridge linear regressor \citep{ridge} to predict latitude and longitude based on the model's representations. We then measure the probe's performance on the test set using the $R^2$ correlation coefficient.

\subsubsection{Results}
In \autoref{fig:maps}, we display the predictions of the probe for the most performant layer, which is generally the last one. We observe that geographical information can be extracted from models even for a very small parameter count. The performance of the probes seem to increase with the model size.

\begin{figure}[h]
    \centering
    \begin{subfigure}[b]{0.37\textwidth}
         \includegraphics[width=\linewidth]{sources/part_1/geographical/imgs/r_square_world_decoders.png}
         \caption{Decoder models}
         \label{fig:decoder_evol}
         \vspace{1em}
    \end{subfigure}
    \begin{subfigure}[b]{0.37\textwidth}
         \includegraphics[width=\linewidth]{sources/part_1/geographical/imgs/r_square_world_encoders.png}
         \caption{Encoder models}
         \label{fig:encoder_evol}
    \end{subfigure}
    \caption{Evolution of the $R^2$ coefficient on the test set for various model suites.}
    \label{fig:evol}
\end{figure}

We show in \autoref{fig:evol} that the performance of language models evolves consistently with model size, regardless of the architecture. We validate this property on several decoder model families: GPT-2 \citep{gpt2}, OPT \citep{zhang2022opt}, Pythia \citep{pythia}, GPT-Neo \citep{gpt-neo}, the multilingual mGPT \citep{shliazhko2023mgpt}, and Llama-2 \citep{touvron2023llama}. We also display results for several encoder models: BERT \citep{devlin-etal-2019-bert,turc2020wellread}, RoBERTa \citep{roberta}, ELECTRA \citep{electra}, and DeBERTa-v3 \citep{deberta}. This property also applies for encoder models, for which we notice that the BERT suite unexpectedly outperforms its counterparts. The performance of encoder models is comparable with the one of equivalent decoder models. We can underline the fact that BERT-Large (336M parameters) is as accurate as the three times larger Pythia-1B.

Interestingly, the multilingual XLM-R \citep{conneau2020unsupervised} underperforms its counterparts, even though multilingual data must have increased the training data's geographical diversity to some extent \citep{faisal-anastasopoulos-2021-investigating}. The mGPT suite also slightly underperforms Pythia models at equivalent model sizes.

We verified that the better performance of larger models was not solely related with the ability of the probes to extract better patterns from their higher-dimensionality hidden representations. We achieved this by concatenating representations with themselves to increase dimensionality without introducing novel knowledge. It led to slightly worse performance for all tested models, thus showing that performance was not a consequence of dimensionality alone.

\subsection{Geographical Bias and Scale}

In \autoref{fig:maps}, it seems at first glance that as the model size increases, the predictions tend to be more accurate for locations of the Southern Hemisphere. In this section, we propose to quantify this hypothesized behavior by measuring the bias across countries and continents for various scales. We also correlate the models' accuracy with both lexical and geographical factors.

\subsubsection{Measuring bias}

We group probe performance as measured by mean-squared error (MSE) on predicted coordinates, and average measures by continent in \autoref{fig:continent_perf}. While we notice that the performance increases consistently for every continent, we do not observe a significant reduction in the performance gap across continents as model size increases.

\begin{figure}
    \centering
    \includegraphics[width=0.9\linewidth]{sources/part_1/geographical/imgs/mse_continent.png}
    \caption{Average MSE by continent for different sizes in the Pythia suite.}
    \label{fig:continent_perf}
\end{figure}

To measure the heterogeneity of the probing performance of language models across countries, we use the Gini coefficient \citep{gini1912variabilita} that is widely used in economics. Given a series of observed variables $(x_i)_{i\in[1, N]}$, the Gini coefficient is defined as:
$$
Gini(x) = \frac{\sum_{i,j \in [1, N]} |x_i - x_j|}{N \cdot \sum_{i = 1}^{N} x_i}
$$

A Gini coefficient of 1 reflects perfect heterogeneity, while a Gini of 0 implies perfect homogeneity.

\begin{figure}
    \centering
    \includegraphics[width=0.75\linewidth]{sources/part_1/geographical/imgs/gini_combined.png}
    \caption{Gini coefficients of MSE on the test set averaged by country or by continent, as model size increases.}
    \label{fig:ginis}
\end{figure}

\autoref{fig:ginis} shows that the larger the model is, the more heterogeneous the probe performance is across countries and continents. This contradicts the impression given by \autoref{fig:maps}, and shows that scale does not solve the geographical discrepancy caused by bias inherent to the training data.

\begin{figure}
    \centering
    \includegraphics[width=0.9\linewidth]{sources/part_1/geographical/imgs/heatmap_100_50.png}
    \caption{Test log-MSE for Pythia-1B as plotted on a World map.}
    \label{fig:heatmap}
\end{figure}

In \autoref{fig:heatmap}, we locally average log-MSE on a World map, and report results agglomerated according to latitude and longitude. We clearly observe that the model performs poorly in Oceania, South Asia and South America. We also see that the error is minimal around the latitude of North America and Europe, while it increases in the Southern Hemisphere.

\subsubsection{Identifying sources of bias}
We attempt to correlate the performance of our geographical probes with several factors. First, the dataset from \citep{gurnee2023language} provides each location with an estimate of the corresponding population count when relevant. We also consider training data distribution as a potential factor of heterogeneity. Finally, we consider latitude and longitude as potential factors of bias.


To account for training data distribution, we look for exact string matches of country names from the \citet{gurnee2023language} dataset in an extract of The Pile \citep{gao2020pile} containing 3.5 million samples \footnote{\url{https://huggingface.co/datasets/ola13/small-the\_pile}}. We select this dataset as it was used to pretrain the models from the Pythia suite \citep{pythia} we evaluate in this section. We find 15 million matches, covering 98\% of the countries of the dataset.

We do not count occurrences of location names directly, as matching locations on the basis of their names does not account for named entity ambiguity. An example of ambiguous location name is \textit{Fully}, which is a town in Switzerland. An exact match strategy overestimates by large margins the occurrence count of this location, because of the corresponding English word \textit{fully}. Disambiguation techniques have been designed \citep{hoffart-etal-2011-robust, orr2020bootleg}, but we prefer to avoid the risk of bias propagation and the cost of using such methods on a large corpus.

\begin{figure}
    \centering
    \includegraphics[width=0.9\linewidth]{sources/part_1/geographical/imgs/correl_size.png}
    \caption{Pearson correlation coefficients of various factors with location-wise MSE, for several Pythia model sizes. *: Tests that yielded p-values above 0.05.}
    \label{fig:correl}
\end{figure}

We display Pearson correlations between each of the aforementioned factors and the entity-level MSE for each model size in \autoref{fig:correl}. As in \autoref{fig:14m_map}, we observe that the error on coordinates prediction is negatively correlated with the latitude, i.e. southern locations are less accurately identified. This correlation slowly decays as the model size increases. Meanwhile, longitude seems to be mildly correlated with the probe performance.

Interestingly, the population count is not correlated with the error level. The occurrence count of the location country is negatively correlated with the error level, thus showing that the more country names appear in the training dataset, the more the probes are able to recover coordinates from locations in these countries. However, this correlation is mild and even below the significance threshold for the smallest model.

We also measure the correlation between country occurrences and other metrics to account for the bias inherent to the data. We observe that country name occurrences are positively correlated with latitude with a p-value of 0.06, and not correlated with the longitude. More importantly, the population count of a country and the count of this country name in the data are heavily correlated (factor of +0.52 and p-value of 3e-23). Thus, even though the data seems guided by demographic factors, this is not the case of the model's representations.

\subsection{Discussion}
\label{sec:discussion}
We believe that quantifying sociocultural bias in representations of language models and pretraining datasets allows to better understand the roots of the biases that can be observed during generation.

\citet{parrots_bender} discuss the relevance of scaling models to ever larger magnitudes, with regard to environmental and financial costs. Our study shows that scale can also increase language modeling bias when it comes to geographical representation, given a pretraining dataset. We advocate in favor of measuring and mitigating bias in pretraining datasets to avoid scaling bias along with performance.

\subsection*{Conclusion}
In this study, we show that a wide variety of language models, varying in architecture and sizes, implicitly embed geographical data to some extent. As we consider larger models, the performance of geographical probes consistently increases towards levels shown in \citet{gurnee2023language}.

We show numerically that the geographical probe performance is correlated with latitude across model sizes, but also with the number of occurrence of corresponding country names in the pretraining data. Conversely, the population count of the location seems uncorrelated with the probe performance. This indicates that a minority of people benefit from better geographical understanding when using language models, which does not maximize the social utility of these systems.

While it may initially seem that this performance increase mitigates heterogeneity between Southern and Northern countries, we actually show that larger models tend to be more biased according to the Gini coefficient taken on prediction error. This tends to show that scaling language models can amplify discrepancies in their geographical knowledge.




% \section*{Acknowledgements}
% This work was funded by the last authors' chair in the PRAIRIE institute, funded by the French national agency ANR as part of the ``Investissements d'avenir'' program under the reference ANR-19-P3IA-0001.
% We thank Stella Biderman for her insightful advice.

% % We would like to thank Roman Castagné for useful discussions that led to focusing on observing the effect of anisotropy in the self-attention process.

% % Entries for the entire Anthology, followed by custom entries
% \section*{Bibliographical References}
% \label{sec:reference}
% \bibliography{sources/part_1/geographical/lrec-coling2024-natbib}
% \bibliographystyle{sources/part_1/geographical/lrec-coling2024-natbib}

% \section*{Language Resource References}
% \label{lr:ref}
% \bibliographylanguageresource{sources/part_1/geographical/lrec-coling2024-natbib}
% \bibliographystylelanguageresource{sources/part_1/geographical/lrec-coling2024-natbib}

% \clearpage

% \appendix


%%% Local Variables:
%%% mode: latex
%%% TeX-master: t
%%% End:


\chapter{Studying Language Model Saturation via the Softmax Bottleneck}
\label{chap:softmax_bottleneck}

\subsection{Introduction}
The representation degeneration problem is a common phenomenon that affects self-supervised learning methods used for textual data \citep{gao2018representation,lai-etal-2023-mitigating}, among other modalities \citep{jing2022understanding,godey2024anisotropy}.
% 
This phenomenon consists in the emergence of degenerated structures in the intermediate latent spaces of language models throughout training.
In particular, many observations on the intermediate representations of Language Models (LMs) have shed light on their low angular variability (or \textit{anisotropy}) by showing that cosine-similarity between pairs of intermediate embeddings tend to be unexpectedly high \citep{freq-based-dist, rajaee-pilehvar-2022-isotropy}. Other works have identified outlier dimensions that emerged during training \citep{puccetti-etal-2022-outlier}. However, these observations were mostly made on relatively small-scale models of dimensions comparable to BERT \citep{devlin-etal-2019-bert} or models from the GPT-2 family \citep{radford2019language}.

These models are usually composed of a neural network $f_\theta$ that takes sequences of tokens $(y_{<i}) \in [1,V]^{i-1}$ as inputs and produces a relatively low-dimensional contextual representation in $\mathbb{R}^d$, where $d$ is the \textit{hidden dimension} of the model. They then rely on a \textit{language modeling head} that produces logits for contextual token probabilities. A common choice for the language modeling head is a linear layer with parameter $W \in \mathbb{R}^{V \times d}$, where $V$ is the number of possible tokens. The resulting next-token probability distribution is then given by:
$$
p(y_i) = \sigma (W f_\theta(y_{<i}))
$$
where $\sigma$ is the softmax function.

In the language modeling field, the current trend consists in scaling up the generative pretraining approach introduced with GPT-2, which implies training neural models made of several billions of parameters on gigantic web-mined text corpora \citep{brown2020language, touvron2023llama, almazrouei2023falcon, jiang2023mistral}. However, training and serving such highly parameterized models raises energy and hardware-related problematics, which motivates for looking into achieving similar performance levels with smaller models \citep{beyond_chinchilla}.

Nevertheless, the evaluation of the Pythia model suite \citep{biderman2023pythia} has shown that training small models on very large corpora could lead to \textit{saturation}, in the form of a performance degradation in late pretraining. In this paper, we explore this saturation phenomenon through the lens of representation degeneration, and find that both phenomena strongly correlate. We further demonstrate that representation degeneration strongly occurs in the language modeling head of small models, and we theoretically and empirically show how a linear language modeling head can represent a performance bottleneck for architectures based on small hidden dimensions.

Overall, our contributions can be summarized as:
\begin{itemize}
    \item We characterize the performance saturation of small language models through evaluation and extrapolation of the scaling laws;
    \item We find that the representations of smaller models degenerate concurrently with this saturation. We shed light on \textit{rank saturation}, i.e. the explosion of the entropy of singular value distributions of small LM prediction heads;
    \item We empirically verify that the rank of the target contextual distribution is usually
    high. Moreover, we observe that regardless of the expressiveness of the output
    representations of a model, a linear head $W$ substantially affects performance when
    $rank(W) < 1000$;
    \item We theoretically quantify the performance limitation induced by a low-rank linear language modeling head.
\end{itemize}

% Our observations identify a bottleneck in causal language modeling that harms the optimization process of small language models.

\subsection{Related Works}
\paragraph{Small LMs \& Saturation} \citet{biderman2023pythia} train Pythia, a suite of models of various sizes on 300B tokens from the Pile \citep{gao2020pile}, and release the weights for an exhaustive number of checkpoints during training. They notice that smaller models suffer a performance decrease on the Lambada dataset \citep{lambada} in late training. The scaling laws \citep{kaplan_scaling,chinchilla_scaling} predict that training smaller models on large corpora is suboptimal in terms of compute. However, recent initiatives \citep{tinyllama,faysse2024croissantllm,gemmateam2024gemma} have pretrained smaller language models on large datasets, motivated by inference cost reduction \citep{beyond_chinchilla}.

\paragraph{Softmax Bottleneck} The concept of \textit{softmax bottleneck} was introduced in \citet{softmax_bottleneck}, where the authors show that a model using a hidden dimension inferior to the rank of the contextual probability matrix cannot predict correctly in every context. They then hypothesize that this rank is relatively high in natural language and propose an alternative method for the predictive layer of language models. Subsequent works have explored negative effects of the softmax linear layer on language modeling performance \citep{chang-mccallum-2022-softmax} and possible alternatives \citep{lin2021breaking,sigsoftmax}. We extend this line of work by quantifying the critical dimensionalities involved in the softmax bottleneck.

\paragraph{Representation Degeneration} is a phenomenon in which pretrained models tend to adopt low-entropy singular value distributions \citep{jing2022understanding}. In language modeling, representation degeneration takes the form of anisotropy \citep{ethayarajh-2019-contextual, rajaee-pilehvar-2021-cluster} and was proven to be related with the Zipfian shape of token distribution \citep{gao2018representation,bis-etal-2021-much}. We study this phenomenon along training and its relation with saturation.

\paragraph{Data Dimensionality and Performance} \citet{scaling_manifold} link the scaling laws observed across pretrained models to data dimensionality, through the lens of Intrinsic Dimension \citep{intrinsic_d}. While they show that Singular Value Decomposition (SVD) is not suited for studying the dimensionality of the data manifold in the universal approximation paradigm, we argue that it is well-suited, to a certain extent, when studying the performance of a linear classifier limited by the dimensionality of input representations.


\subsection{Language Model Saturation}
We first verify that we can indeed observe and quantify performance saturation for the Pythia checkpoints, as they are the only released intermediate checkpoints for a wide range of model sizes. We measure the cross-entropy of Pythia checkpoints on 50k tokens randomly sampled from their pretraining dataset, i.e. The Pile \citep{gao2020pile}. 

\begin{figure}[h]
\centering
    \begin{subfigure}{0.37\columnwidth}
         \includegraphics[width=\linewidth]{sources/part_1/softmax_bottleneck/imgs/loss_saturation_anno.pdf}
         \caption{Loss saturation}
         \label{fig:loss_sat}
    \end{subfigure}
    \begin{subfigure}{0.42\columnwidth}
         \includegraphics[width=\linewidth]{sources/part_1/softmax_bottleneck/imgs/scaling_laws_unfit_anno.pdf}
         \caption{Loss extrapolation}
         \label{fig:scaling_law}
    \end{subfigure}
    \caption{Performance of Pythia models on the Pile. On the left, we compare training dynamics of models from 14M (top) to 410M (bottom) parameters, displaying darker shades as we approach the minimal value. On the right, we fit a power law on larger models and find that final checkpoints of smaller models underperform compared to predictions.}
    \label{fig:saturation}
\end{figure}
In \Cref{fig:loss_sat}, we clearly see that models up to 410M parameters suffer from the saturation phenomenon, characterized as an increase of the in-domain loss in advanced training stages. 

In \Cref{fig:scaling_law}, we fit a scaling law in the style of \citet{chinchilla_scaling} on data points from models ranging from 410M parameters, only optimizing for model-related constants ($A$ and $\alpha$) while reusing all other values ($B=410.7$, $\beta=0.28$, $E=1.69$). We recall the relation between parameter count $N$ and token count $T$ given in \citet{chinchilla_scaling}:
$$
L(N, T) = \frac{A}{N^\alpha} + \frac{B}{T^\beta} + E
$$

We find that optimal parameters are $A=119.09$ and $\alpha=0.246$. We display the fitted curves for token counts that correspond to best and final checkpoints. We observe that the final checkpoints underperform the extrapolation by 8\% in average. The loss-minimizing (\textit{best}) checkpoints, which are expected to fall short of the extrapolation due to their incomplete learning rate cooldown, only underperform it by roughly 4\%.

A similar performance saturation is also observed on datasets used for evaluation in the LM Evaluation Harness \citep{eval-harness}, as shown in \Cref{tab:perf_gap}.

\begin{table}[h]
% \scriptsize
\centering
\scalebox{0.85}{%
\begin{tabular}{@{}lcccccc@{}}
\toprule
\multicolumn{1}{c}{Checkpoint} & Lambada (ppl.) $\downarrow$ & Lambada $\uparrow$ & StoryCloze $\uparrow$ & WikiText (ppl.) $\downarrow$ & SciQ $\uparrow$ & ARC-e $\uparrow$\\ \midrule
Best & \textbf{24.6} & \textbf{40.3} & \textbf{59.6} & \textbf{30.47} & \textbf{79.6} & \textbf{46.5} \\
Final & 32.9 & 38 & 57.2 & 33.4 & 73.4 & 43.2 \\ \bottomrule
\end{tabular}}
\caption{Zero-shot performance of Pythia-160M best and final checkpoints on evaluation datasets. Unless specified, we report accuracy for all tasks.}
\label{tab:perf_gap}
\end{table}


\subsection{Performance Saturation is Rank Saturation}
\subsubsection{Anisotropy at Scale}
Anisotropy is a common form of representation degeneration that has been observed among various small language models. It consists in reduced angular variability of the representation distribution at a given layer. Previous works \citep{ethayarajh-2019-contextual,godey2024anisotropy} notice that almost all layers of small Transformers language models are anisotropic. A common measure for anisotropy in a set $H$ of vector representations is the average cosine-similarity:
$$
\mathcal{A}(H) = \frac{1}{|H|^2 - |H|} \sum_{h_i, h_j \in H, i \neq j} \frac{h_i^Th_j}{||h_i||_2 \cdot ||h_j||_2}
$$

However, it remains unclear whether anisotropy affects models with over 1 billion parameters. In order to address this question, we compute average cosine-similarity of intermediate representations across layers in suites of models; namely GPT-2 \citep{radford2019language}, OPT \citep{zhang2022opt}, Pythia \citep{biderman2023pythia}, and Gemma \citep{gemmateam2024gemma}. We use a subsample of The Pile \citep{gao2020pile}, as we hypothesize that the domain of this dataset includes or matches the domain of the pretraining datasets used in these suites.

\begin{figure}[h]
    \centering
    \begin{subfigure}{0.33\columnwidth}
         \includegraphics[width=\linewidth]{sources/part_1/softmax_bottleneck/imgs/pythia_anisotropy.png}
         \caption{Pythia}
         \label{fig:pythia_aniso}
    \end{subfigure}
    \begin{subfigure}{0.33\columnwidth}
         \includegraphics[width=\linewidth]{sources/part_1/softmax_bottleneck/imgs/gpt2_anisotropy.png}
         \caption{GPT-2}
         \label{fig:gpt2_aniso}
    \end{subfigure}
    \begin{subfigure}{0.33\columnwidth}
         \includegraphics[width=\linewidth]{sources/part_1/softmax_bottleneck/imgs/gemma_anisotropy.png}
         \caption{Gemma}
         \label{fig:gemma_aniso}
    \end{subfigure}
    \begin{subfigure}{0.34\columnwidth}
         \includegraphics[width=\linewidth]{sources/part_1/softmax_bottleneck/imgs/opt_anisotropy.png}
         \caption{OPT}
         \label{fig:opt_aniso}
    \end{subfigure}
    \caption{Anisotropy in function of layer depth (i.e. order in the forward pass).}
    \label{fig:anisotropy}
\end{figure}

In \Cref{fig:anisotropy}, we observe that most layers of Transformers models are anisotropic to some extent, regardless of the scale. Nevertheless, there seems to be a dichotomy in the last layer, where models are either nearly isotropic or highly anisotropic. Interestingly, we notice that the dichotomy aligns with the one of the saturation phenomenon for the Pythia suite, where only models containing 160M or fewer parameters seem affected by last-layer anisotropy.

We thus decide to study the training dynamics of anisotropy for the Pythia suite, and compare them with the saturation phenomenon in \Cref{fig:aniso_v_saturation}.

\begin{figure}[h]
    \centering
    \begin{subfigure}{0.32\columnwidth}
         \includegraphics[width=\linewidth]{sources/part_1/softmax_bottleneck/imgs/anisotropy_explosion_14m.png}
         \caption{14M}
         \label{fig:14M}
    \end{subfigure}
    \begin{subfigure}{0.32\columnwidth}
         \includegraphics[width=\linewidth]{sources/part_1/softmax_bottleneck/imgs/anisotropy_explosion_31m.png}
         \caption{31M}
         \label{fig:31M}
    \end{subfigure}
    \begin{subfigure}{0.32\columnwidth}
         \includegraphics[width=\linewidth]{sources/part_1/softmax_bottleneck/imgs/anisotropy_explosion_70m.png}
         \caption{70M}
         \label{fig:70M}
    \end{subfigure}
    \begin{subfigure}{0.32\columnwidth}
         \includegraphics[width=\linewidth]{sources/part_1/softmax_bottleneck/imgs/anisotropy_explosion_160m.png}
         \caption{160M}
         \label{fig:160M}
    \end{subfigure}
    \begin{subfigure}{0.33\columnwidth}
         \includegraphics[width=\linewidth]{sources/part_1/softmax_bottleneck/imgs/anisotropy_explosion_410m.png}
         \caption{410M}
         \label{fig:410M}
    \end{subfigure}
    \caption{Evolution of the language modeling performance on the Wikipedia test set from the LM Evaluation Harness \citep{eval-harness} and last-layer anisotropy of Pythia models along training (color).}
    \label{fig:aniso_v_saturation}
\end{figure}

\Cref{fig:aniso_v_saturation} illustrates a neat correlation between the emergence of the performance saturation phenomenon and the appearance of anisotropy in the last-layer representations of the models. It also shows that anisotropy increases abruptly around the saturation point during training. Moreover, we see here that on a specific in-domain corpus, the models quickly lose performance at saturation and never seem to fully recover from this explosion.

\subsubsection{Singular Values Saturation}
\label{sub:saturation}

Average cosine-similarity is a valuable measure of the uniformity of a distribution, but including other metrics can help to better capture the complexity of some manifolds \citep{rudman-etal-2022-isoscore}. Moreover, it only focuses on the output embeddings of the language models, and not on their weights. In this section, we extend our analysis by studying the singular value distributions of the language modeling heads, to link our empirical observations to our theoretical findings. In \Cref{fig:sv_evolve}, we display the singular value distributions of the final predictive layer weights $W$ along training.

\begin{figure}[h]
    \centering
    \begin{subfigure}{0.32\columnwidth}
         \includegraphics[width=\linewidth]{sources/part_1/softmax_bottleneck/imgs/sv_map_14m.png}
         \caption{14M}
         \label{fig:sv_14M}
    \end{subfigure}
    \begin{subfigure}{0.32\columnwidth}
         \includegraphics[width=\linewidth]{sources/part_1/softmax_bottleneck/imgs/sv_map_31m.png}
         \caption{31M}
         \label{fig:sv_31M}
    \end{subfigure}
    \begin{subfigure}{0.32\columnwidth}
         \includegraphics[width=\linewidth]{sources/part_1/softmax_bottleneck/imgs/sv_map_70m.png}
         \caption{70M}
         \label{fig:sv_70M}
    \end{subfigure}
    \begin{subfigure}{0.32\columnwidth}
         \includegraphics[width=\linewidth]{sources/part_1/softmax_bottleneck/imgs/sv_map_160m.png}
         \caption{160M}
         \label{fig:sv_160M}
    \end{subfigure}
    \begin{subfigure}{0.32\columnwidth}
         \includegraphics[width=\linewidth]{sources/part_1/softmax_bottleneck/imgs/sv_map_410m.png}
         \caption{410M}
         \label{fig:sv_410M}
    \end{subfigure}
    \caption{Evolution of the singular value distributions of the LM heads of Pythia models during training, normalized by the maximum singular value.}
    \label{fig:sv_evolve}
\end{figure}

\Cref{fig:sv_evolve} sheds light on a specific pattern of spectral saturation, roughly co-occurring with the performance saturation phenomenon. It shows that the singular value distribution progressively flattens during training, and nearly reaches uniformity before abruptly evolving towards a spiked distribution with a high maximal singular value, relatively to the other ones.

\begin{wrapfigure}{r}{0.45\textwidth}
\centering
    \includegraphics[width=0.45\textwidth]{sources/part_1/softmax_bottleneck/imgs/kullback_uni.png}
    \caption{Training dynamics of the singular entropy, for different Pythia models.}
    \vspace{-10pt}

    \label{fig:kl_div}
\end{wrapfigure}

In order to quantify this behavior more accurately, we use a \textit{singular entropy metric}, computed as the Kullback-Leibler divergence between the normalized singular value distribution and the uniform distribution.

\Cref{fig:kl_div} shows that singular distributions evolve differently for models using less than 410M parameters than for the larger ones. The heads of small models see their singular value distributions become increasingly uniform, up to a point where they degenerate abruptly, which again correlates with the LM performance drop. The singular value distributions of larger models tend to be more stable, and do not display clear monotonic patterns throughout training.

\subsection{The Softmax Bottleneck \& Language Dimensionality}
\subsubsection{Inherent Dimensionality of Natural Language}
\label{sec:inherent_dim}
% In practice, we can neither access $W^*$ nor $\phi^*$. However, we propose to use proxies for both in order to estimate the magnitude of the dimensionality of natural language in terms of singular spectrum.
Intuitively, the saturation of the singular values distribution observed only for smaller models in \Cref{sub:saturation} questions the dimensionalities involved in the optimization of the LM head. In this section, we propose to empirically measure a critical value for the rank of the LM head, and to estimate the dimensionality of the contextual probability distribution the head's outputs are supposed to match.

In order to empirically measure the effect of the rank of the linear head, we propose to train rank-constrained heads on pretrained contextual representations from highly-parameterized language models. In order to control the maximum rank $r$, we consider heads of the form $W = AB \in \mathbb{R}^{V \times d}$, where the coefficients of $A \in \mathbb{R}^{V \times r}$ and $B \in \mathbb{R}^{r \times d}$ are drawn from $\mathcal{N}(0, 1)$ ($d$ being the hidden dimension of the model). The rank of such $W$ matrices is limited by the parameter $r \in [1, d]$, which we sweep over a wide range of values.

We freeze the language models and train the rank-constrained heads on their output representations on roughly 150M tokens, while adjusting the learning rate to the trainable parameter count (more details in \Cref{app:hyperparams}).

\begin{figure}[h]
\centering
    \begin{subfigure}{0.41\columnwidth}
         \includegraphics[width=\linewidth]{sources/part_1/softmax_bottleneck/imgs/llama_bottleneck_acc.png}
         \caption{Accuracy}
         \label{fig:bottleneck_acc}
    \end{subfigure}
    \begin{subfigure}{0.42\columnwidth}
         \includegraphics[width=\linewidth]{sources/part_1/softmax_bottleneck/imgs/llama_bottleneck_loss.png}
         \caption{Cross-entropy}
         \label{fig:bottleneck_ce}
    \end{subfigure}
    \caption{Performance of several models as the bottleneck dimension of the head increases.}
    \label{fig:perf_bottleneck}
\end{figure}
In \Cref{fig:perf_bottleneck}, we observe that perplexity starts to noticeably decrease when the rank of the language modeling head $W$ is inferior to 1000, \textit{regardless of the model size}. This hints that the head is not a major performance bottleneck for models with greater hidden dimensions, but that it may hurt performance for models with smaller ones independently of the quality of the output representations.

Another interesting factor to estimate is the dimensionality inherent to the data itself. To avoid possible effects related to specific inductive biases, we train naive 5-gram language models on several datasets of varying coverage (IMDb \citep{imdb}, Wikitext \citep{wikitext}, and The Pile \citep{gao2020pile}), using two tokenizers of varying vocabulary sizes (30k tokens for Llama-2 and 50k tokens for Pythia). Given $C$ observed 5-grams, we consider the matrices $W \in \mathbb{R}^{C \times V}$ where each row is a probability distribution over possible tokens in a given 4-token context, and compute their singular value distributions, as in \citet{ngram_svd}. In \Cref{fig:w_error}, we report \textit{$W$-error}, the minimal approximation error on $W$ for a matrix of rank $d$ as predicted by the Eckart-Young-Mirsky theorem (see \Cref{eym}), normalized by the Frobenius norm of $W$:
$$
W\text{-error}(d) = \frac{||\sigma_{d+1:}||_2}{||W||_F}
$$
\begin{figure}[h]
\centering
    \begin{subfigure}{0.415\columnwidth}
         \includegraphics[width=\linewidth]{sources/part_1/softmax_bottleneck/imgs/llama_sv_4gram.png}
         \caption{Llama-2 tokenizer}
         \label{fig:llama}
    \end{subfigure}
    \begin{subfigure}{0.4\columnwidth}
         \includegraphics[width=\linewidth]{sources/part_1/softmax_bottleneck/imgs/pythia_sv_4gram.png}
         \caption{Pythia tokenizer}
         \label{fig:pythia}
    \end{subfigure}
    \caption{$W$-error as $d$ increases, for different tokenizers and datasets. We observe that while W-error can be halved using 1000 or 2000 dimensions, it only becomes negligible after 10,000-15,000 dimensions.}
    \label{fig:w_error}
\end{figure}

We find that the estimated rank of $W$ is non-negligible with respect to the usual magnitude of hidden dimensions. In the next section, we analyze the connection between the dimensionality of an ideal linear language modeling head and performance from a theoretical perspective.


\subsubsection{A Theoretical Bottleneck}
In this section, we aim at identifying a formal link between the inherent dimensionality of the contextual distribution and the performance bottleneck that can be attributed to the lower dimensionality of the output representations of a language model. To that end, we conceptualize a language modeling head optimized on \textit{ideal} contextual representations, and we explore the relationship between its spectral properties and the performance gap induced when training a low-rank head on the same representations.  

% Let's consider a set $\mathcal{T}$ of sequences $(y_i)_{i \in [1, |y|]}$ of elements taken from a vocabulary of size $V$, representing the pretraining data. We consider a function $\phi^*_D$ that \textit{perfectly} (e.g. in a bijective way) represents a given context $y_{< i}$ as a single real vector of \textit{sufficiently} high dimension $D \in \mathbb{N}^* \cup \{+\infty\}$. As we do not focus on $\phi^*_D$, we can simplify the notations by introducing the contextual representations $x^*_i = \phi^*_D(y_{< i})$. 

Let's consider a set $\mathcal{T}$ of sequences $(y_i)_{i \in [1, |y|]}$ of elements taken from a vocabulary of size $V$, representing the pretraining data. We consider a function $\phi^*$ that \textit{perfectly} (e.g. in a bijective way) represents a given context $y_{< i}$ as a single real vector of \textit{infinite} dimension. As we do not focus on $\phi^*$, we can simplify the notations by introducing the contextual representations $x^*_i = \phi^*(y_{< i})$. 

The task of the linear language modeling head can be formalized as an optimization problem on the matrix $W$:
\begin{equation}
W^* = \argmin_{W \in \mathbb{R}^{V \times \infty}} \sum_{y \in \mathcal{T}} \sum_{i} \mathcal{L}(W, x^*_i, y_i)
\label{eq:unconstrained}
\end{equation}

where $\mathcal{L}$ is the cross-entropy objective defined using the softmax function $\sigma$ as:
$$
\mathcal{L}(W, x, y) = - \log (\sigma(Wx)_{y})
$$

% Without loss of generality, we choose $D$ to be large enough so that setting up this problem using $D+1$ would not lead to a better performance in \autoref{eq:unconstrained}.

In practice, a neural language model $\phi_{\theta}$ produces contextual representations $x_i = \phi_{\theta}(y_{< i})$ of dimension $d \in \mathbb{N}^*$. The linear language modeling head $W_\theta \in \mathcal{R}^{V \times d}$ is trained concurrently with $\phi_{\theta}$ with the same objective as in \autoref{eq:unconstrained}.

We focus on the maximal expressiveness of a lower-dimensional head: when provided with \textit{perfect} contextual representations $x^*_i$, what is the maximal performance level of a linear language modeling head of maximal rank $d$? This question can be put in mathematical terms:

\begin{equation}
W_d^* = \argmin_{W \in \mathbb{R}^{V \times \infty}} \sum_{y \in \mathcal{T}} \sum_{i} \mathcal{L}(W, x^*_i, y_i) \text{ s.t. } rank(W) \leq d
\label{eq:constrained}
\end{equation}

% MADE THINGS COMPLICATED with x*:
% \Cref{best_on_all} shows that any alternative to the $W^*$ matrix, including a low-rank approximation, cannot improve performance in any context.


% \begin{lemma}
% \label[lemma]{best_on_all}{(proof in \Cref{app:best_on_all})}
% By construction of $W^*$, for all $W \in \mathbb{R}^{V \times \infty}$, $y \in \mathcal{T}$ and $x^* = \phi^*(y)$, we have:
% $$
% \mathcal{L}(W, x^*, y) \geq \mathcal{L}(W^*, x^*, y)
% $$
% \end{lemma}


\Cref{linear_rel} shows that by approaching $W^*$ directly, we can asymptotically expect to close the performance gap.

\begin{lemma}
\label[lemma]{linear_rel}{(proof in \Cref{app:linear_rel})}
Let's consider $W \in \mathbb{R}^{V \times \infty}, M \in \mathcal{H}^{V \times \infty}$ the matrix unit sphere for the Frobenius norm $||\cdot||_F$, and $\varepsilon \in \mathbb{R}^*_+$ such that $W = W^* + \varepsilon M$ . When $\epsilon \rightarrow 0$:
$$
|\mathcal{L}(W, x^*_i, y_i) - \mathcal{L}(W^*, x^*_i, y_i)|  = O(\varepsilon)
$$
\end{lemma}

% \begin{lemma}
% \label[lemma]{linear_rel}{(proof in \Cref{app:linear_rel})}
% Let's consider $W \in \mathbb{R}^{V \times \infty}, M \in \mathcal{H}^{V \times \infty}$ the matrix unit sphere for the Frobenius norm $||\cdot||_F$, and $\varepsilon \in \mathbb{R}^*_+$ such that $W = W^* + \varepsilon M$ . When $\epsilon \rightarrow 0$:
% $$
% |\mathcal{L}(W, x^*_i, y_i) - \mathcal{L}(W^*, x^*_i, y_i)|  = K_{W^*, x^*, y} \cdot \varepsilon + o(\varepsilon)
% $$
% \end{lemma}

Hence, our problem is linked to a low-rank matrix approximation \citep{low_rank}, which has direct connections with spectral theory. In our case, we can use the Eckart–Young–Mirsky theorem.

\begin{lemma}
\label[lemma]{eym}{(Eckart–Young–Mirsky theorem)}
Let's consider $(\sigma_i)$ the singular values of $W^*$ in decreasing order, and $\mathcal{M}_d$ the set of matrices in $\mathbb{R}^{V \times \infty}$ of rank $d < V = rank(W^*)$. Then:
$$
\min_{W_d \in \mathcal{M}_d}||W_d - W^*||_F = \sqrt{\sum_{i=d+1}^{V} \sigma_i^2}
$$
\end{lemma}

Combining all of the above yields \Cref{main_thm}.
% Stronger but weaker
% \begin{lemma}
% \label[lemma]{linear_rel}{(proof in \Cref{app:linear_rel})}
% Let's consider $W \in \mathbb{R}^{D \times V}$, $M \in \mathcal{H}^{D \times V}$ the matrix unit sphere for the Frobenius norm $||\cdot||_F$, and $\epsilon \in \mathbb{R}^*_+$ such that $W = W^* + \epsilon M$ . When $\epsilon \rightarrow 0$, there is a constant $K_{W^*, x^*_i, M} \geq 0$ such that:
% $$
% |\mathcal{L}(W, x^*_i, y_i) - \mathcal{L}(W^*, x^*_i, y_i)| \geq \epsilon K_{W^*, x^*_i, M} + o(\epsilon)
% $$
% \end{lemma}

%From \Cref{linear_rel}, we know that minimizing $||W_d^* - W^*||_F$ also minimizes \autoref{eq:constrained}.%

\begin{theorem}{(proof in \Cref{app:main_thm})}
\label{main_thm}
Let's consider $(\sigma_i)$ the singular values of $W^*$ in decreasing order. Then, when $d \rightarrow V$, the loss gap induced by a $d$-dimensional bottleneck on the linear LM head follows:
$$
\sum_{y \in \mathcal{T}} \sum_{i} \mathcal{L}(W_d^*, x^*_i, y_i) - \mathcal{L}(W^*, x^*_i, y_i) = O\left(\sqrt{\sum_{i=d+1}^{V}\sigma_i^2}\right)$$
\end{theorem}

These properties shed light on how the dimensionality of the ideal language modeling head impacts the performance when the LM head is low-rank. However, the relation obtained in \Cref{main_thm} is not particularly strong, as discussed in \Cref{app:main_thm}.

In \Cref{fig:neg_res_thm}, we compare the results of the head bottleneck experiment of the Pythia-1B model in \Cref{sec:inherent_dim} to the $W$-error on the head of the same model as the bottleneck dimension $d$ evolves. It shows that the loss gap grows slowly with the $W$-error, implying that even when the allowed rank would lead to a poor approximation of $W$, the performance can still remain acceptable. We notice that the performance starts decreasing when the $W$-error outgrows 0.6.

\begin{wrapfigure}{r}{0.45\textwidth}
\centering
    \includegraphics[width=0.45\textwidth]{sources/part_1/softmax_bottleneck/imgs/loss_v_werr.png}
    \caption{Final loss with trained rank-constrained heads (mimicking $W_d^*$), as a function of the theoretical $W$-error for rank $d$ on the head of the Pythia-1B model.}
    \vspace{-10pt}

    \label{fig:neg_res_thm}
\end{wrapfigure}

% \begin{figure}[h]
% \centering
%     \includegraphics[width=0.4\linewidth]{sources/part_1/softmax_bottleneck/imgs/loss_v_werr.png}
%     \caption{Final loss with trained rank-constrained heads (mimicking $W_d^*$), as a function of the theoretical $W$-error for rank $d$ on the head of the Pythia-1B model.}
%     \label{fig:neg_res_thm}
% \end{figure}

% This is mostly due to the peculiar nature of the target distributions, which make a direct low-rank approximation to $W^*$ particularly suboptimal in terms of cross-entropy loss. 

% Moreover, contexts are far from uniformly distributed, which means that the optimal overall low-rank performance can be further optimized by focusing on the more frequent contexts. A better frame for this problem would be weighted low-rank approximation \citep{srebro2003weighted}; however, such an optimization problem does not have a closed form to the best of our knowledge. We leave exploration in this direction for future work.



% \subsubsection{The Role of Unigram Frequency}
% We have shown that the LM head of a small model tends to adopt a degenerated state in late training, which is correlated with the low performance of the resulting model. In this section, we propose to study this degenerated state at representation level, in order to understand what it implies for the model behavior and why it may lead to suboptimal performance.

% \textcolor{blue}{
% What will be here?
% \begin{itemize}
%     \item a figure showing how the norm of the average output representation increases in training, and how it aligns with frequency ($\sigma(W\bar{x}) \rightarrow f$)
%     \item conclusion: frequency becomes more important during training
%     \item empirical examples of high frequency terms appearing too often in generation after explosion vs. before
% \end{itemize}
% }
\newpage
\subsection{Discussion}

% This paper provides insights on the performance bottleneck represented by linear language modeling heads, stressing out that the dimensionality of the target contextual distribution is poorly matched by language models that rely on small hidden dimensions.

One way to address the problem at hand could be to train shallow small language models, increasing hidden dimension at the expense of other hyperparameters, such as layer count or feed-forward dimension. However, we believe that such research directions may not be promising in this context. Previous works have extensively explored and optimized the hyperparameter choices for various architecture sizes. The impact of width and depth has been extensively studied \citep{merrill-etal-2022-saturated, tay2022scale, petty2023impact}, often showcasing the importance of depth in final performance and generalization capabilities.

Another possible way forward consists in implementing more expressive softmax alternatives \citep{softmax_bottleneck,chang-mccallum-2022-softmax} in the context of pretraining small language models on large datasets. We leave the exploration of such techniques for future work.

We also believe that further exploration of the specific nature of the singular components after the collapse we describe in \Cref{sub:saturation} could improve our understanding of LM saturation. We hypothesize that the resulting dominating components are correlated with token frequency, based on previous works that link anisotropy with token frequency \citep{gao2018representation,ethayarajh-2019-contextual,bis-etal-2021-much} and show the importance of token frequency in the LM head mechanism \citep{meister-etal-2023-natural}.

Beyond the scope of this article, we argue that our work demonstrates that last-layer anisotropy is symptomatic of performance saturation, and is thus likely not a desirable property of language models. We also advocate that this work paves the way towards a better understanding of the structure of the contextual probability distribution, which could also enhance our interpretation of the scaling laws.

\section*{Conclusion}
Small language models can be affected by performance saturation during training. We find that this phenomenon can be explained by an inherent difficulty in mapping a low-dimensional output representation space to a high-rank contextual probability distribution through a linear language modeling head. Indeed, we show a theoretical link between the performance gap induced by a smaller hidden dimension and the spectral properties of the contextual probability distribution.

We empirically confirm that the rank of such a mapping can be expected to be relatively high compared to regular hidden dimension choices. Moreover, we conduct experiments to measure the impact of constraining the rank of the LM head on the performance of a large language model. Our results show that performance noticeably drops when using a hidden dimension smaller than roughly 1000. We further analyze the saturation phenomenon through the lens of spectral analysis and find that the emergence of last-layer anisotropy that only affects small models can be correlated with saturation. We also show that the LM heads of small models concurrently suffer from \textit{spectral} saturation, i.e. a uniformization of singular values that leads to a degenerated state.

Our work paves the way for a better understanding of the consequences of the softmax bottleneck on language modeling, and for the conception of language models that better embrace the complexity of the target probability distribution.

\section*{Limitations}
The main limitation of this article is the relatively small amount of saturated language models we studied. As it is the only suite of language models trained in the range of interest to release an extensive amount of intermediate checkpoints, we could only observe the training dynamics of small Pythia models. Although we observe strong last-layer anisotropy for the smallest GPT-2 model, we cannot tell with certainty whether it suffered from saturation. The OPT-125m model does not display a strong last-layer anisotropy, which could indicate that it was not affected by the saturation phenomenon.

Nevertheless, we argue that this paper does not show that \textit{all} small models should suffer from saturation, but rather that the saturation of small language models is symptomatic of a limitation that may affect language models that are based on a relatively small hidden dimension. Furthermore, we do not state that there is a causality relationship between degeneration and low hidden dimension choices, but rather expose a strong correlation between both phenomenon that can be explained through the prism of our softmax bottleneck analysis.


Another limitation of this work is the loose nature of the mathematical connection that we establish between the dimensionality of the ideal language modeling head and the rank-constrained performance (cf. \Cref{main_thm}). Moreover, it can also be argued that considering \textit{ideal} $x_i^*$ representations is an ill-defined notion. We argue that the reasoning behind \Cref{main_thm} could be applied to any contextual representations, as the \textit{ideal} nature of $x_i^*$ is not necessary in the demonstrations. The word \textit{ideal} reflects that our observations hold for $x_i^*$ representations obtained from \textit{any underlying model}, to an extent that depends on the structure that these representations impose on the $W^*$ matrix for a given training set $\mathcal{T}$.

\section*{Acknowledgements}
We thank Song Duong for carefully reviewing this article and for his valuable suggestions.

This work was funded by the last author's chair in the PRAIRIE institute funded by the French national agency ANR as part of the ``Investissements d'avenir'' programme under the reference ANR-19-P3IA-0001. 

This work was granted access to the HPC resources of GENCI operated by IDRIS (allocation 2023-AD011013680R1).


% Previous works \citep{ethayarajh-2019-contextual,godey2024anisotropy} notice that almost all layers of Transformers language models are anisotropic. They notably show that the last layer of decoder models display extremely high anisotropy metrics. For instance, \citet{ethayarajh-2019-contextual} show that the representations of the last layer of the small version of GPT-2 \citep{radford2019language} have an average cosine-similarity of 0.97.

% The matter of outlier dimensions has been studied for larger language models, notably in the quantization literature \citep{bondarenko2023quantizable,lee2024owq}.
% % 







% \subsection{Theoretical contributions}

% \subsubsection{Low-rank approximation}
% We 

% \subsubsection{Optimization problem}

% Language modeling with a linear classifier can also be thought of as an optimization problem for a matrix mapping context to token probability, in the style of.
% \begin{equation} \label{eq1}
% \begin{split}
% \min_{W\in\mathcal{M}^d} E_{c,y}(\mathcal{L}(W, c, y)) & = \frac{1}{|c|} \sum_{c \in \mathcal{D}} \sum_y P^*(y|c)\log P_W(y|c)\\
% & = \sum_{c \in C} P(c)\sum_y P^*(y|c)\log P_W(y|c) \\
% & = \sum_{c \in C}\sum_y (P(c)P^*(y|c))\log P_W(y|c) \\
% \end{split}
% \end{equation}

% $$
% \min_W E_{c,y}(\mathcal{L}(W, c, y)) = 
% $$
% $$
% \min_W E_{c,y}(\mathcal{L}(W, c, y)) = \sum_{c \in C} P(c)\sum_y P^*(y|c)\log P_W(y|c)
% $$
% % $$
% % \mathcal{L}(W^* + \epsilon, x^*_i, y_i) - \mathcal{L}(W^*, x^*_i, y_i) = -(\epsilon x^*_i)_{y_i} + \log\left(\frac{\sum_j{e^{{(W^*x^*_i)}_j} e^{{(\epsilon x^*_i)}_j}}}{\sum_j{e^{({W^*x^*_i})_j}}}\right)
% % $$

% % $$
% % \mathcal{L}(W^* + \epsilon, x^*_i, y_i) - \mathcal{L}(W^*, x^*_i, y_i) \geq -(\epsilon x^*_i)_{y_i} + \frac{\sum_j{e^{{(W^*x^*_i)}_j} {(\epsilon x^*_i)}_j}}{\sum_j{e^{({W^*x^*_i})_j}}}
% % $$


% \subsubsection*{Author Contributions}
% If you'd like to, you may include  a section for author contributions as is done
% in many journals. This is optional and at the discretion of the authors.

% \subsubsection*{Acknowledgments}
% Use unnumbered third level headings for the acknowledgments. All
% acknowledgments, including those to funding agencies, go at the end of the paper.


\subsection{Proofs}
% \subsubsection{\Cref{best_on_all}}
% \label{app:best_on_all}
% This proof can be done by contradiction. Let's suppose that there exist $W \in \mathbb{R}^{V \times \infty}$, $y \in T$ and $x^*=\phi^*(y)$, and $i \in [1, |y|]$ such that $\mathcal{L}(W, x^*_i, y_i) < \mathcal{L}(W^*, x^*_i, y_i)$. We can build a block diagonal matrix $W_{2\infty} = \begin{bmatrix}
% W^* \\ W
% \end{bmatrix}$ along with the $x_{2\infty}$ representations as $\begin{bmatrix}
% \phi^*(T \setminus \{y_i\}) & 0 \\
% 0 & \phi^*(y_i)
% \end{bmatrix}$.

% It is clear that for all $z \in T \setminus \{y_i\}$ and all $j \in [1, |z|]$, we have $\mathcal{L}(W^*, \phi^*(z_j), z_j) = \mathcal{L}(W_{2\infty}, \phi^*(z_j), z_j)$, and $\mathcal{L}(W_{2\infty}, x^*_i, y_i) = \mathcal{L}(W, x^*_i, y_i) < \mathcal{L}(W^*, x^*_i, y_i)$. Thus, $W_{2\infty}$ is a better minimizer of the problem defined in \autoref{eq:unconstrained}, which contradicts the definition of $W^*$.


\subsubsection{\Cref{linear_rel}}
\label{app:linear_rel}
The proof is mainly based on calculations and limited development:
\begin{align*}
& |\mathcal{L}(W, x^*_i, y_i) - \mathcal{L}(W^*, x^*_i, y_i)| \\
& =
\left| -\log \frac{\exp((Wx^*_i)_{y_i})}{\sum_{j \in V} \exp((Wx^*_i)_{j})} + \log\frac{\exp((W^*x^*_i)_{y_i})}{\sum_{j \in V} \exp((W^*x^*_i)_{j})}\right| \\
& =  \left|-(\varepsilon Mx^*_i)_{y_i} + \log \frac{\sum_{j \in V} \exp((W^*x^*_i)_{j}) \exp((\varepsilon Mx^*_i)_{j})}{\sum_{j \in V} \exp((W^*x^*_i)_{j})}\right| \\
& = \left| - \varepsilon (M x^*_i)_{y_i} + \log\left( 1 + \frac{\sum_{j \in V} \varepsilon \exp((Mx^*_i)_{j})}{\sum_{j \in V} \exp((W^*x^*_i)_{j})} + o(\varepsilon) \right) \right| \\
& = \left| -\varepsilon (M x^*_i)_{y_i} + \varepsilon  \frac{\sum_{j \in V} \exp((Mx^*_i)_{j})}{\sum_{j \in V} \exp((W^*x^*_i)_{j})} \right| + o(\varepsilon) \\
& = \varepsilon \left| - (M x^*_i)_{y_i} + \frac{\sum_{j \in V} \exp((Mx^*_i)_{j})}{\sum_{j \in V} \exp((W^*x^*_i)_{j})} \right| + o(\varepsilon)
\end{align*}

The continuous function $M \longrightarrow \left| - (M x^*_i)_{y_i} + \frac{\sum_{j \in V} \exp((Mx^*_i)_{j})}{\sum_{j \in V} \exp((W^*x^*_i)_{j})} \right|$ is bounded on the compact matrix unit sphere (i.e. where $||M||_F = 1$), which ends the proof.

\textbf{Remark : }This result could also be proven using a differentiability argument, but we prefer to display a more precise relation between the loss gap and the error on the $W$ matrix approximation, stressing out its quasi-linear nature. This formulation will hopefully pave the way for further exploration of this relation in future works.

\subsubsection{\Cref{main_thm}}
\label{app:main_thm}
Let us note $W_d$ the best approximation of $W^*$ of rank $d$ with respect to the Frobenius norm. By definition of $W_d^*$, we have that:
\begin{equation}
\label{eq:approx}
    \left|\sum_{y \in \mathcal{T}} \sum_{i} \mathcal{L}(W_d^*, x^*_i, y_i) - \mathcal{L}(W^*, x^*_i, y_i)\right| \leq \sum_{y \in \mathcal{T}} \sum_{i} \left|\mathcal{L}(W_d, x^*_i, y_i) - \mathcal{L}(W^*, x^*_i, y_i)\right|
\end{equation}

% We know from \Cref{best_on_all} that for all $y\in\mathcal{T}$ and $i \in [1, |y|]$:
% $$
% \mathcal{L}(W_d, x^*_i, y_i) - \mathcal{L}(W^*, x^*_i, y_i) \geq 0
% $$

The Eckart-Young-Mirsky theorem tells us that when $d \rightarrow V$, 
$$||W_d - W^*||_F = \sqrt{\sum_{i=d+1}^{V} \sigma_i^2} \rightarrow 0$$

By defining $\varepsilon = W_d - W^*$, we can apply \Cref{linear_rel} and show that:
$$
\left|\mathcal{L}(W_d, x^*_i, y_i) - \mathcal{L}(W^*, x^*_i, y_i)\right| = O(||W_d - W^*||_F) = O\left(\sqrt{\sum_{i=d+1}^{V} \sigma_i^2}\right)
$$

From \Cref{eq:approx}, we have that:
$$
\left|\sum_{y \in \mathcal{T}} \sum_{i} \mathcal{L}(W_d^*, x^*_i, y_i) - \mathcal{L}(W^*, x^*_i, y_i)\right| = O\left(\sqrt{\sum_{i=d+1}^{V} \sigma_i^2}\right)
$$

By definition of $W^*$ and $W_d^*$, we also have that:
$$
0 \leq \sum_{y \in \mathcal{T}} \sum_{i} \mathcal{L}(W_d^*, x^*_i, y_i) - \mathcal{L}(W^*, x^*_i, y_i) = \left|\sum_{y \in \mathcal{T}} \sum_{i} \mathcal{L}(W_d^*, x^*_i, y_i) - \mathcal{L}(W^*, x^*_i, y_i)\right|
$$
which ends the proof.

\paragraph{Remark} The bound used in \Cref{eq:approx} can be rather loose in practice. We can think of no particular reason why approaching $W^*$ directly should be the optimal way to minimize the loss on $\mathcal{T}$. Hence, the presented result should be taken carefully, and we leave the refinement of such an analysis for future work.

% Jensen's inequality for concave functions can be applied to the $\log$ term:
% \begin{align*}
% \mathcal{L}(W, x^*_i, y_i) - \mathcal{L}(W^*, x^*_i, y_i) &\geq -(\epsilon x^*_i)_{y_i} + \frac{\sum_{j \in V} \exp((W^*x^*_i)_{j}) (\epsilon x^*_i)_{j}}{\sum_{j \in V} \exp((W^*x^*_i)_{j})} \\
% &\geq ||\epsilon||_F \min_{M \in \mathcal{H}^{D \times N}} -(M x^*_i)_{y_i} + \frac{\sum_{j \in V} \exp((W^*x^*_i)_{j}) (M x^*_i)_{j}}{\sum_{j \in V} \exp((W^*x^*_i)_{j})} \right)
% \end{align*}

% where $\mathcal{H}^{D \times N}$ is the matrix hypersphere, i.e. $\{M \in \mathbb{R}^{D \times N} s.t. ||M||_F = 1 \}$.


% \subsection{Eckart–Young–Mirsky theorem}
% \label{app:eym}
% \begin{lemma}
% \label[lemma]{eym}{(Eckart–Young–Mirsky)}
% Let's consider $(\sigma_i)$ the singular values of $W^*$ in decreasing order, and $\mathcal{M}_d$ the set of matrices of rank $d < D$. Then:
% $$
% \min_{W_d \in \mathcal{M}_d}||W_d - W^*||_F = \sqrt{\sum_{i=d+1}^{D} \sigma_i^2}
% $$
% \end{lemma}

\subsection{Hyperparameters}
\label{app:hyperparams}

\subsubsection{Constrained head experiments (\Cref{fig:perf_bottleneck})}

We freeze the pretrained weights in the Transformer layers, and we train each rank-constrained head (i.e. in the form $W=AB$ with $r$ as the inner dimension of the matrix product) for various values of $r$ on 150M tokens sampled from The Pile using 4 V100 GPUs for the Pythia models and 4 A100 GPUs for Llama-7B. We use the hyperparameters from \citet{biderman2023pythia}, except for the batch size which we set to 256 as it fits our hardware setup better. As the trainable parameter count evolves with $r$, we search for the best-performing learning rates among values ranging from $1\cdot 10^{-3}$ to $5\cdot 10^{-2}$.

We report the chosen learning rates in \Cref{fig:lr_choices}.

\begin{figure}[h]
\centering
    \includegraphics[width=0.6\linewidth]{sources/part_1/softmax_bottleneck/imgs/lr_final.png}
    \caption{Chosen peak learning rates used for the rank-constrained head experiments for each model.}
    \label{fig:lr_choices}
\end{figure}




\chapter{Anisotropy Is Inherent to Self-Attention in Transformers}

\subsection{Introduction}
In recent years, deep learning models based on Transformers have led to significant breakthroughs in the field of natural language processing (NLP). These models have demonstrated state-of-the-art performance across a range of tasks, such as language modeling, machine translation, and sentiment analysis. However, despite their successes, they suffer from a phenomenon known as the representation degeneration problem. Specifically, this degeneration is characterized by anisotropy, a property of hidden representations that makes them all close to each other in terms of angular distance (cosine-similarity).

Anisotropy has been widely observed among self-supervised models based on Transformers, and literature currently suggests that it may be a consequence of optimizing the cross-entropy loss on long-tailed distributions of tokens \citep{gao2018representation, bis-etal-2021-much}. However, it remains uncertain whether anisotropy is a fundamental property of Transformers-based models or a consequence of the pre-training process.

In this paper, we investigate the anisotropy problem in depth, and we make several contributions:
\begin{itemize}
    \item We demonstrate empirically that anisotropy can be observed in language models with character-aware architectures that should not suffer directly from the same consequences as token-based models. We extend our observations to Transformers trained on other modalities, such as image and audio data, and show that anisotropy cannot be explained solely based on linguistic properties;
    \item We provide empirical observations on the anisotropic properties of the Transformer block by studying untrained layers, and establish a relation between anisotropy and the general sharpness of the self-attention mechanism;
    \item We conduct an analysis of the representations used in self-attention (queries and keys) along training and show that anisotropy appears intrinsically in the self-attention mechanism, when training pushes for sharp patterns.
\end{itemize} 

\subsection{Related Work}


\begin{figure}[ht]
    \centering
     \includegraphics[width=0.9\columnwidth]{sources/part_1/anisotropy/imgs/cosine_token.png}
     \caption{Average cosine-similarity between hidden representations across layers for token-level NLP models. For T5-base, we concatenate encoder and decoder results.}
     \label{fig:anisotropy_token}
\end{figure}

The general phenomenon of anisotropy in token-based Transformers for language models has been shown in \citet{ethayarajh-2019-contextual}. \autoref{fig:anisotropy_token} extends one of their experiment to more architectures. \citet{gao2018representation} shows that the degeneration of representations comes from the distributions of subwords in natural language, namely the existence of unused and rare tokens that tend to push all representations away from the origin towards a specific direction.

Other works have established a connection between word frequency and distortions of the latent spaces \citep{yu-etal-2022-rare, puccetti-etal-2022-outlier, rajaee-pilehvar-2022-isotropy}. \citet{bis-etal-2021-much} have shown that anisotropy in LMs could be explained by a global \textit{drift} of the representations in the same direction, thus unifying conclusions from \citet{ethayarajh-2019-contextual} and \citet{gao2018representation}. The authors propose that this drift is caused by the persistent updating of the representation of rare and unused tokens in a consistent direction, due to the nature of the softmax operation in the cross-entropy loss. They show that removing the average component to all representations leads to a nearly perfect isotropy.

Several methods have been proposed to reduce anisotropy in Transformers-based LMs at token-level \citep{rajaee-pilehvar-2021-cluster, Wang2020Improving}, or at sentence-level \citep{gao-etal-2021-simcse, yan-etal-2021-consert, su2021whitening}. They usually consist in post-processing the representations, and lead to downstream performance boosts. We argue that these positive results are paving the way for the search of pre-training objectives that do not introduce anisotropy in the first place, in the hope that the resulting models will also perform better without any post-processing, and potentially be trained more efficiently. This motivates us to gain a deeper understanding of the underlying factors that induce anisotropy, whether they belong in data, architectures, or training procedures.

\subsection{Anisotropy in pre-trained Transformers}
\subsubsection{Character-based NLP}
\label{sec:charbased}
\begin{figure}[ht]
    \centering
     \includegraphics[width=0.9\columnwidth]{sources/part_1/anisotropy/imgs/cosine_char_based.png}
     \caption{Average cosine-similarity between hidden representations across layers for character-level models.}
     \label{fig:cos_char_aware}
\end{figure}

To assert whether the cross-entropy objective applied on vocabularies containing rare tokens is the sole cause for the common drift issue, we explore anisotropy in character-based models. We study different architectures:
\begin{itemize}
    \item CharacterBERT \citep{el-boukkouri-etal-2020-characterbert} is constructing whole word representations from character embeddings put through convolutions and highway layers, before feeding them to a Transformers architecture.
    \item CANINE \citep{clark-etal-2022-canine} is downsampling contextualized character representations via a strided convolution before feeding them to a Transformers. It can be trained either with a subword-based objective (CANINE-s) or with a character-level one (CANINE-c).
    \item MANTa-LM \citep{godey-etal-2022-manta} is based on a differentiable segmentation and embedding module added before an encoder-decoder model in the style of T5 \citep{2020t5}. It takes bytes as inputs and outputs, but builds internal representations that are usually based on several bytes.
    \item ByT5 \citep{xue-etal-2022-byt5} is a version of T5 that is trained at byte-level. To afford for more complex encoding, the authors resize the encoder-decoder architecture.
\end{itemize}

Neither of these architectures should suffer from out-of-vocabulary tokens in the process of creating representations. The models that predict at word or sub-word level (CharacterBERT and CANINE-s) could have the cross-entropy loss systematically pushing away rare item representations. However, it is rather unclear why it would imply an embedding drift at deeper layers. Hence, if anisotropy was only caused by the presence of unused or rare subwords, those character-level models should be much less prone to this issue.

To verify this hypothesis, we compute hidden representations for the validation set of the WikiText-103 corpus \citep{merity2017pointer}. We then compute the average cosine-similarity between two representations, uniformly taken in the whole validation corpus.

In fact, as shown in \autoref{fig:cos_char_aware}, those models all display significant levels of anisotropy in at least one of their layers. Interestingly, the models that are based solely on characters or bytes for input and prediction (ByT5, CANINE-c, and MANTA-LM) seem to display even higher levels of anisotropy. We note, as it is the case for the T5 model, that the ByT5 decoder displays extremely high levels of anisotropy.

\subsubsection{Other modalities}
\label{sec:other_mod}
\begin{figure*}[ht]
    \centering
    \begin{subfigure}[b]{0.43\textwidth}
         \includegraphics[width=\linewidth]{sources/part_1/anisotropy/imgs/cosine_audio.png}
         \caption{Speech}
         \label{fig:cos_speech}
    \end{subfigure}
    \hfill
    \begin{subfigure}[b]{0.43\textwidth}
         \includegraphics[width=\linewidth]{sources/part_1/anisotropy/imgs/cosine_vit_imagenet.png}
         \caption{Vision}
         \label{fig:cos_audio}
    \end{subfigure}
    \caption{Average cosine-similarity between hidden representations across layers for Speech and Vision modalities. We observe that across both modalities, several models display significant levels of anisotropy.}
    \label{fig:anisotropy_modalities}
\end{figure*}

We've shown in the previous section that character-level language models suffer from anisotropy similarly to token-level ones, hinting that subword token distributions are not solely responsible for anisotropy. However, it may be argued that anisotropy is related to linguistic properties. Thus, we proceed to explore the anisotropy problem for Transformers-based models in other modalities, specifically speech and vision.

For speech models, we consider wav2Vec 2.0 \citep{wav2vec}, HuBERT \citep{HuBERT}, and Whisper \citep{radford2022whisper} with the Common Voice 11.0 dataset \citep{commonvoice:2020}. For vision models, we use ViT \citep{Wu2020VisualTT}, BEiT \citep{beit-2021}, MiT \citep{segformer21}, and DEiT \citep{pmlr-v139-touvron21a} on the ImageNet dataset \citep{imagenet15russakovsky}.

As in \autoref{sec:charbased}, we infer hidden representations on the validation sets for each modality. We then uniformly sample pairs of vectors to get cosine-similarity values for every layer of every model. The averaged results are displayed in \autoref{fig:anisotropy_modalities}.

Once again, almost every model shows a significant level of anisotropy on some of its layers. Notably, speech models seem to have very anisotropic representations, as every layer of every model outputs an average cosine-similarity of at least $0.2$. We find some exceptions among vision models, since the MiT model seems to use isotropic representation spaces and the ViT model has a low average cosine-similarity for all its layers.

We also conduct the same experiment for convolution-based networks in the vision modality. The models at glance are ResNet \citep{he2016deep}, EfficientNet \citep{Tan2019EfficientNetRM}, CvT \citep{wu2021cvt}, ConvNeXt \citep{liu2022convnet}, and VAN \citep{guo2022visual}. For these networks, we flatten convolution maps to vectors before computing the cosine-similarity.

\begin{figure}[ht]
    \centering
    \includegraphics[width=\linewidth]{sources/part_1/anisotropy/imgs/cosine_cnn_imagenet.png}
    \caption{Average cosine-similarity between hidden representations across layers for convolution-based vision models.}
    \label{fig:convbased}
\end{figure}

We observe in \autoref{fig:convbased} that most of the convolution-based models are isotropic. Interestingly, the only exception is ResNet-50, whose representations become more and more isotropic as one explores deeper layers. This could partially be explained by the fact that the batch normalization \citep{pmlr-v37-ioffe15} used in some of these models mitigates \textit{a posteriori} the drift effect by removing the mean component of the representations. However, the ConvNeXt model also seems to use isotropic representations while not using batch normalization, which shows that this is not the only factor in the isotropic behavior of these models.

\subsubsection{To drift or not to drift?}
Related works \citep{bis-etal-2021-much, gao2018representation} show that anisotropy in subword-level language models is caused by a drift of the hidden representations in a shared direction. In this section, we try to extend this observation to other modalities.

We study the correlation between the uniformly measured cosine-similarity, and the norm of the average hidden representation $||\bar{x}||_2$ for each layer. If anisotropy could be directly explained by the drift effect, we would expect a monotonic relation between $||\bar{x}||_2$ and the average cosine-similarity. To verify this, we apply a Spearman correlation test on these two metrics for every model from \autoref{sec:charbased} and \autoref{sec:other_mod}, along with some token-level language models, namely T5 \citep{2020t5}, BERT \citep{devlin-etal-2019-bert}, RoBERTa \citep{roberta}, and GPT-2 \citep{gpt2}.

\begin{figure}[ht]
    \centering
    \includegraphics[width=0.9\linewidth]{sources/part_1/anisotropy/imgs/pval_vs_cosine_spearman.png}
    \caption{p-value of the Spearman correlation test between the norm of the average representation and the cosine-similarity averaged over all layers, across modalities. For models above the red dotted line, there is no significant ($p>0.05$) correlation between the drift effect and the anisotropy level.}
    \label{fig:pval_vs_cos}
\end{figure}

In \autoref{fig:pval_vs_cos}, we observe that we can correlate the anisotropy level and the magnitude of the drift component across layers for several models. The anisotropy of subword-based models can generally be correlated with the drift effect, except for GPT-2 for which the Spearman correlation metric may not be appropriate. We provide a similar analysis based on the Pearson correlation test and discuss the relevance of each statistic in \Cref{app:pearson}. 

Interestingly, we notice that the anisotropy affecting most CNN-based vision models is generally not correlated with the drift effect, contrary to Tranformers-based models in the same modality. Some speech models (HuBERT and Whisper-base) also display signs of anisotropy that cannot be correlated with the drift effect. \autoref{fig:pval_vs_cos} also shows a correlation for all character-based models but Canine-C and MANTa-base.

\subsection{Exploring the representation drift}
\label{sec:empirical}
% We've empirically explored the extent of the anisotropy effect in Transformers-based models, and showed that it affects various modalities.
In this section, we focus on some intrinsic properties of the Transformer block in a modality-agnostic fashion, i.e. with minimal assumptions on the data distribution, and without training. We analyze experimentally the behavior of the untrained Transformer block $T$ when a common bias term $b$ is added to untrained input representations $\mathbf{x}$. This allows us to mimic the common drift as mentioned in \citet{bis-etal-2021-much} and to identify some properties induced by this artificial drift on the output representations.

\subsubsection{Experimental setup}
We consider an embedding lookup table $E$ and a Transformer block $T$ with weights initialized as in BERT \citep{devlin-etal-2019-bert}. We then draw 16 input embedding sequences $\mathbf{x}$ of length 512 uniformly from $E$. To account for a drift component of norm $N\in\mathbb{R}$, we generate a vector $b_u \sim \mathcal{N}(0, I_d)$, which we normalize into $b = \frac{b_u}{||b_u||_2}\times N$. We finally compute $T(\mathbf{x}_i + b)$ for every sequence $x_i$, and study the resulting distributions.

Specifically, we study the average norm of the input representations $\mathbb{E}(||\mathbf{x}_i + b||_2)$ against the average norm of the output representations $\mathbb{E}(||T(\mathbf{x}_i + b)||_2)$ in \autoref{fig:norm_scratch_transformer}. We also retrieve the self-attention scores before the softmax operation, namely $\frac{QK^T}{\sqrt{d_k}}$, along with the corresponding $Q$ and $K$ matrices. We study some of their properties in \autoref{fig:attscore_trained_transformer} and \autoref{fig:kq}.

\subsubsection{Input vs. output analysis}
\begin{figure}[ht]
    \centering
    \begin{subfigure}[b]{0.8\columnwidth}
         \includegraphics[width=\linewidth]{sources/part_1/anisotropy/imgs/scratch_bert_base_input_vs_output.png}
         \subcaption{Cosine similarity}
         \label{fig:cos_scratch_transformer}
         
    \vspace{1.2em}
    \end{subfigure}
    \begin{subfigure}[b]{0.8\columnwidth}
         \includegraphics[width=\linewidth]{sources/part_1/anisotropy/imgs/bert_base_norm_v_output.pdf}
         \subcaption{Norm}
         \label{fig:norm_scratch_transformer}
    \end{subfigure}
    \caption{Input/Output comparison of a Transformer block from BERT-base as the bias norms increases.}
    \label{fig:bias_vs_cosine_norm}
\end{figure}

In \autoref{fig:cos_scratch_transformer}, we observe that the output representations have an average cosine-similarity value that is slightly higher than the one of the input representations, no matter the level of input bias. We also notice that while the norm of the average output representation increases with the bias norm, it seems to meet the corresponding input measure for a given bias norm.

Interestingly, this shows that there is a \textit{fixed point} in terms of norm in the Transformers function with biased input. More formally, there seems to exist a bias norm $N^* \in \mathbb{R}_+$ such that: $$\mathbb{E}_{x, b_{N^*}}(||x_i + b_{N^*}||) = \mathbb{E}_{x, b_{N^*}}(||T(x_i + b_{N^*})||)$$

Moreover, this fixed point level $N^*$ is in the order of magnitude of the average hidden state norms of the layers of the trained BERT model. This hints that the model's representations stabilize when their norm is close to this fixed point. We leave a more thorough analysis of this hypothesis for future work.

\subsubsection{Exploring the Transformer block}

To understand the effect of the drift effect on the inner workings of the Transformer layer, we take a closer look at the self-attention operation as the average input representation drifts away.

\begin{figure}[ht]
    \centering
    \includegraphics[width=0.9\linewidth]{sources/part_1/anisotropy/imgs/trained_bert_base_att_scores.pdf}
    \caption{Histograms of the pre-softmax attention scores as the input bias norm increases. Other initializations of the layer and of the bias direction $b_u$ led to a general \textit{increase} of the attention scores instead.}
    \label{fig:attscore_trained_transformer}
\end{figure}

\autoref{fig:attscore_trained_transformer} shows that the attention scores tend to move away from zero as the input bias norm increases. Indeed, as the norm of the average $\bar{x}$ of the input embeddings increases, we can expect the query and key vectors $Q$ and $K$ to also display signs of anisotropy. Actually, for each self-attention head, and for all position $i \in [1, L]$, we have:
\begin{equation}
    \begin{cases}
      \mathbb{E}_x(Q_i) = W_Q\bar{x} + b_Q\\
      \mathbb{E}_x(K_i) = W_K\bar{x} + b_K
    \end{cases}
\end{equation}

We can observe in \autoref{fig:kq} that query and key representations indeed increase in norm with the input bias norm. We also notice that the corresponding distributions are anisotropic even when no bias is added, which may be a consequence of BERT's initialization parameters.

\begin{figure}[ht]
    \centering
    \begin{subfigure}[b]{0.48\columnwidth}
         \includegraphics[width=\linewidth]{sources/part_1/anisotropy/imgs/trained_bert_base_bias_vs_kq_cos.png}
         \caption{Cosine sim.}
         \label{fig:cos_qk_trained_transformer}
    \end{subfigure}
    \begin{subfigure}[b]{0.48\columnwidth}
         \includegraphics[width=\linewidth]{sources/part_1/anisotropy/imgs/trained_bert_base_bias_vs_kq_norm.png}
         \caption{Norm}
         \label{fig:norm_qk_trained_transformer}
    \end{subfigure}
    \caption{Analysis of the self-attention query and key distributions}
    \label{fig:kq}
\end{figure}

\subsubsection{Impact of the drift}

After exploring the consequences of the drift of input representations on the query-key product in self-attention, we identify in this section the implications of this drift at a more explainable level, by observing the resulting post-softmax distributions.


\begin{figure}[ht]
    \centering
    \includegraphics[width=0.8\linewidth]{sources/part_1/anisotropy/imgs/untrained_bert_base_bias_vs_softmax.png}
    \caption{Evolution of the self-attention softmax values as the input bias norm increases.}
    \label{fig:softmax_trained_transformer}
\end{figure}

In \autoref{fig:softmax_trained_transformer}, we retrieve softmax values in the self-attention block and for each position, we extract the maximum, the median and the minimum. We then average these values over the whole batch, and repeat for various input bias norm levels. We notice that as the input bias norm increases, the self-attention softmax distributions tend to become less entropic, evolving towards higher maximal probabilities and lower minimal probabilities. In the following analysis, we'll use the term \textit{sharpness} to discuss entropy levels of the self-attention distributions.


\begin{figure}[ht]
    \centering
    \begin{subfigure}[b]{0.48\columnwidth}
         \includegraphics[width=\linewidth]{sources/part_1/anisotropy/imgs/untrained_bert_base_bias_vs_max_softmax.png}
         \caption{Maximum}
         \label{fig:max_softmax}
    \end{subfigure}
    \begin{subfigure}[b]{0.48\columnwidth}
         \includegraphics[width=\linewidth]{sources/part_1/anisotropy/imgs/untrained_bert_base_bias_vs_min_softmax.png}
         \caption{Minimum}
         \label{fig:min_softmax}
    \end{subfigure}
    \caption{Comparison of the extreme values of each sequence averaged over the batch as the bias norm increases.}
    \label{fig:min_vs_max}
\end{figure}

This sharpening effect of the attention distributions becomes even clearer if we consider the maximum and minimum values over the whole sequences, as in \autoref{fig:min_vs_max}.

However, at low anisotropy levels, i.e. when the bias norm is low, we see that the effect is not very important. \autoref{fig:softmax_trained_transformer} and \autoref{fig:min_vs_max} only hint at the fact that the drift of embeddings may help the self-attention to be sharper. Another explanation could be that training favors sharp self-attention patterns, as has been pointed out in previous works \citep{clark-etal-2019-bert}, which in turn induces a drift in the models' representations. In order to account for that, we need to study the evolution of latent spaces at the self-attention level along training.

\subsection{Queries and keys: training dynamics}
\label{sec:qk}
\begin{figure*}[ht]
    \centering
    \begin{subfigure}[b]{0.24\linewidth}
         \includegraphics[width=\linewidth]{sources/part_1/anisotropy/imgs/dist_l9h9_s0.png}
         \caption{Step 0}
         \label{fig:dist_qk_s0}
    \end{subfigure}
    \begin{subfigure}[b]{0.24\linewidth}
         \includegraphics[width=\linewidth]{sources/part_1/anisotropy/imgs/dist_l9h9_s40.png}
         \caption{Step 40k}
         \label{fig:dist_qk_s40}
    \end{subfigure}
    \begin{subfigure}[b]{0.24\linewidth}
         \includegraphics[width=\linewidth]{sources/part_1/anisotropy/imgs/dist_l9h9_s200.png}
         \caption{Step 200k}
         \label{fig:dist_qk_s200}
    \end{subfigure}
    \begin{subfigure}[b]{0.24\linewidth}
         \includegraphics[width=\linewidth]{sources/part_1/anisotropy/imgs/dist_l9h9_s2000.png}
         \caption{Step 2M (final)}
         \label{fig:dist_qk_s2M}
    \end{subfigure}
    \caption{Evolution of $Q_s$ and $K_s$ distributions along training. Vectors are projected using a common SVD.}
    \label{fig:proj_qk_heads}
\end{figure*}

We have established that manually pushing for drift-based anisotropy on \textit{untrained} Transformers models leads to sharper (i.e. low-entropy) self-attention patterns. In this section, we show that this evolution of self-attention values actually takes place during training, and we explore the mechanism behind their appearance. As pointed out in \autoref{sec:empirical}, the self-attention scores result from the $QK^T$ operation, which computes scalar products between query and key representations corresponding to each pair of positions. Thus, in this section, we study the evolution of these query and key representations \textit{along training}, and explore the mechanism behind the increase of the scalar products leading to self-attention scores.

We use the MultiBERT checkpoints \citep{sellam2021multiberts} with seed 0 to retrieve $Q$ and $K$ distributions at different pretraining steps, and we use 128 samples from Wikitext-103 as input data. Along this section, $Q_s$ and $K_s$ refer to query and key representations extracted at a specific layer and head at a given step $s$, and $\hat{Q_s}$ and $\hat{K_s}$ are the average representations, taken over all tokens in the sampled batch. By studying $\bar{Q_s}$ and $\bar{K_s}$, we aim at exploring the common (or context-agnostic) drifts of keys and queries distributions.

\begin{figure}[ht]
    \centering
    \begin{subfigure}[b]{0.48\columnwidth}
         \includegraphics[width=\linewidth]{sources/part_1/anisotropy/imgs/l0h3_samedir_QK.png}
         \caption{Similar}
         \label{fig:QK_simdir}
    \end{subfigure}
    \begin{subfigure}[b]{0.48\columnwidth}
         \includegraphics[width=\linewidth]{sources/part_1/anisotropy/imgs/l9h5_diffdir_QK.png}
         \caption{Opposite}
         \label{fig:QK_diffdir}
    \end{subfigure}
    \caption{Evolution of $\bar{Q_s}$ and $\bar{K_s}$ along training for two different heads in the network, projected via common SVD. Each arrow represents a checkpoint in the MultiBERT suite. We display typical examples of dynamics in same/opposite direction.}
    \label{fig:QK_dir}
\end{figure}

In \autoref{fig:proj_qk_heads} and \autoref{fig:QK_dir}, we compute a SVD of the union of $Q_s$ and $K_s$ for all steps $s$, so that the projection makes sense for both distributions across steps for visualization purposes \footnote{We actually uniformly sample 20\% of the whole set of representations to compute the SVD under reasonable memory constraints.}. As shown in our selected examples, we observe that the dynamics of $\bar{Q_s}$ and $\bar{K_s}$ tend to align along training, making the average of the distributions drift in either similar or opposite directions. The first dimension of the SVD seems to describe this common drift. Note that in $\mathbb{R}^{d_h}$ ($d_h = 64$ being the head dimension), such an alignment is very unlikely to happen randomly. Interestingly, \autoref{fig:QK_simdir} shows that the common direction dynamics appear in the first few steps, while the opposite direction dynamics of  \autoref{fig:QK_diffdir} only starts after 8\% of the total training steps.

\begin{figure*}[ht]
    \centering
    \begin{subfigure}[b]{0.24\linewidth}
         \includegraphics[width=\linewidth]{sources/part_1/anisotropy/imgs/l0_cosine_QK.png}
         \caption{Layer 0}
         \label{fig:cosine_qk_l0}
    \end{subfigure}
    \begin{subfigure}[b]{0.24\linewidth}
         \includegraphics[width=\linewidth]{sources/part_1/anisotropy/imgs/l4_cosine_QK.png}
         \caption{Layer 4}
         \label{fig:cosine_qk_l4}
    \end{subfigure}
    \begin{subfigure}[b]{0.24\linewidth}
         \includegraphics[width=\linewidth]{sources/part_1/anisotropy/imgs/l9_cosine_QK.png}
         \caption{Layer 9}
         \label{fig:cosine_qk_l9}
    \end{subfigure}
    \begin{subfigure}[b]{0.24\linewidth}
         \includegraphics[width=\linewidth]{sources/part_1/anisotropy/imgs/l11_cosine_QK.png}
         \caption{Layer 11}
         \label{fig:cosine_qk_l11}
    \end{subfigure}
    \caption{Evolution of cosine-similarity between $\bar{Q_s}$ and $\bar{K_s}$ along training. Each color represents one self-attention head. Steps are counted in thousands. We generally observe that almost all heads see $\bar{Q_s}$ and $\bar{K_s}$ align in common or opposite directions along training. In other words, the average components of keys and queries representations tend to align in self-attention heads, which maximizes the magnitude of the scalar product between two average representations. We run a similar experiment on all MultiBERT seeds in \autoref{fig:seeds_qk}, and obtain comparable results.}
    \label{fig:cosine_qk_heads}
\end{figure*}

To consolidate our observations, we compute the evolution of the cosine-similarity between $\bar{Q_s}$ and $\bar{K_s}$ along training in \autoref{fig:cosine_qk_heads}. We also display some projected $Q_s$ and $K_s$ distributions for several $s$ steps in \autoref{fig:proj_qk_heads}.

\autoref{fig:cosine_qk_heads} shows that the first layers display a common direction dynamic, as the cosine-similarity tends to increase, thus showing that \textbf{the key and query distributions drift along a similar direction} in average. The last layers seem to adopt an opposite direction dynamic, as the cosine-similarity between their mean key and query representations gets negative along training.

\begin{figure}[ht]
    \centering
    \begin{subfigure}[b]{0.48\columnwidth}
         \includegraphics[width=\linewidth]{sources/part_1/anisotropy/imgs/l3h8_scalar_QK.png}
         \caption{Similar}
         \label{fig:scalar_sim}
    \end{subfigure}
    \begin{subfigure}[b]{0.48\columnwidth}
         \includegraphics[width=\linewidth]{sources/part_1/anisotropy/imgs/l9h9_scalar_QK.png}
         \caption{Opposite}
         \label{fig:scalar_opp}
    \end{subfigure}
    \caption{Evolution of the scalar product between $\bar{Q_s}$ and $\bar{K_s}$ along training. Steps are in thousands.}
    \label{fig:scalar_QK}
\end{figure}

As shown in \autoref{fig:scalar_QK}, this drift induces an increase in the magnitude of scalar products obtained in the self-attention $QK^T$ operation, thus facilitating the emergence of sharp patterns where attention focuses on specific tokens.

\begin{figure}[ht]
    \centering
    \includegraphics[width=0.8\linewidth]{sources/part_1/anisotropy/imgs/entropy_decay.png}
    \caption{Average entropy of the probability distributions corresponding to self-attention rows along training. Each curve corresponds to one layer.}
    \label{fig:entropy_decay}
\end{figure}

Finally, \autoref{fig:entropy_decay} describes the evolution of the average entropy in self-attention distributions. We observe that training induces an overall decay of the entropy for all layers, with different dynamics. This corresponds to sharper self-attention distributions. It is interesting to notice that the distributions in the first layers remain sharper than the ones in the last layers.

Overall, this section shows that drift anisotropy emerges in the query and key representations during the training of MultiBERT, as self-attention distributions become sharper. The drifts of queries and keys tend to align, thus increasing the magnitude of scalar products, and the general sharpness of self-attention.

Although this section focuses on the case of token-based NLP, we believe that strong attention patterns may be required when training Transformers across all modalities, potentially generating distortions in query and key distributions that account for the final observed anisotropy of the models. However, we could not extend experiments to other modalities due to the lack of released intermediate checkpoints, to the best of our knowledge.

\subsection{Discussion}
\label{sec:discussion}

In this work, we argue that the nature of data distributions is not solely responsible for the anisotropy observed in most hidden representations of Transformers-based models across modalities. As \autoref{sec:empirical} shows, untrained Transformers layers display a tendency towards anisotropy. Biased inputs tend to increase the variance of the attention scores and thus facilitate the emergence of sharp patterns in the self-attention mechanisms. We also show in \autoref{sec:qk} that along training, query and key distributions drift in parallel directions, which increases anisotropy in the inner representations of the Transformer layers, while allowing sharper attention patterns. As discussed in \citet{puccetti-etal-2022-outlier}, outlier dimensions in Transformers are also involved in the emergence of strong attention patterns.

\paragraph{Consistency of the SVD} In \autoref{sec:qk}, we use an SVD on the \textit{union} of $Q_s$ and $K_s$ for visualization purposes (see \autoref{fig:proj_qk_heads} and \autoref{fig:QK_dir}). It may be argued that this approach favors the emergence of a discriminative singular direction, that helps distinguish between keys and queries, thus supporting the findings in a less convincing way. To address this concern, we display alternative projections in \autoref{sec:other_projs}, where we compute the SVD on $Q_s$ or $K_s$ only, and then project all representations using this SVD. Our observations show that our findings are consistent for these alternative projections.

\paragraph{Harmfulness of anisotropy} Even though anisotropy has not been shown to be an issue in language modeling, previous works have advocated that removing anisotropy in output representations leads to better sense disambiguation abilities \citep{bihani-rayz-2021-low, bis-etal-2021-much}. Isotropic models could also improve cross-lingual alignment in multilingual language models \citep{hämmerl2023exploring}. Nevertheless, concurrent works have suggested that anisotropy may not hurt the quality of the representations \citep{ait-saada-nadif-2023-anisotropy, rudman2023stable}. We argue that anisotropy in the Transformer architecture may actually help models by allowing sharp attention patterns, but we also believe that our work can pave the way for new architectures that can easily use sharp attention patterns without inducing anisotropy. 


\subsection*{Conclusion}
In this paper, we investigated the anisotropy problem through the lens of the drift effect, and made several contributions to the understanding of this phenomenon. We demonstrated that anisotropy can be observed in language models with character-aware architectures, extended our observations to Transformers trained on other modalities, and studied anisotropy in untrained Transformers layers. We finally explored the training dynamics of the query and key distributions, and found that they drift along a shared direction hence maximizing $QK^T$ scalar products in absolute value, allowing stronger attention patterns as a result.

We conclude that anisotropy almost systematically affects Transformers on all modalities, in a way that is not always correlated with the drift of the representations. We also provide empirical evidence that anisotropy appears as an inherent property of latent distributions used in the self-attention mechanism when modeling sharp attention patterns. We hypothesize that a revision of the self-attention operation could help reduce anisotropy by facilitating the emergence of sharp attention softmax distributions without distorting the geometry of the hidden representations.


\subsection*{Limitations}
As mentioned in the Discussion section, we acknowledge that \autoref{sec:empirical} does not take into account the training dynamics, and only exposes some properties of the Transformer layer at initialization. We also notice that the Spearman correlation test used in \autoref{fig:pval_vs_cos} may not be well-suited for such noisy observations, as the high p-value of the GPT-2 model shows. We provide a similar graph based on the Pearson correlation in \Cref{app:pearson}.

Moreover, we are aware that our approach is not theoretically rigorous in some aspects. For instance, we don't prove that sharp self-attention patterns \textit{cannot} emerge without anisotropy in keys and queries representations. In other words, this article is focusing on exposing and \textit{correlating} factors that explain anisotropy, but we do not demonstrate theoretical properties that would help identify the \textit{causes} of anisotropy. Nevertheless, we believe that our work can pave the way for such theoretical exploration in the future.


\subsection*{Ethics Statement}
To the best of our knowledge, our work does not raise any ethical concern. However, as noted in \citet{zhou2021freqbased}, we believe that distortions in the embedding space may be related to bias in the training data, whether it is inherent to the structure of the modality (e.g. the Zipfian distribution of words), or due to human factors (e.g. geographical considerations).

\subsection*{Acknowledgements}
This work was funded by the last authors' chair in the PRAIRIE institute funded by the French national agency ANR as part of the ``Investissements d'avenir'' programme under the reference ANR-19-P3IA-0001. This work was granted access to the HPC resources of IDRIS under the allocation 2023-AD011013680R1 made by GENCI.

We would like to thank Roman Castagné for useful discussions that led to focusing on observing the effect of anisotropy in the self-attention process.

% Entries for the entire Anthology, followed by custom entries


\subsection{Pearson correlation of the drift norm and anisotropy}
\label{app:pearson}

\begin{figure}[ht]
    \centering
    \includegraphics[width=\linewidth]{sources/part_1/anisotropy/imgs/pval_vs_cosine_pearson.png}
    \caption{p-value of the Pearson correlation test between the norm of the average representation and the cosine-similarity averaged over all layers, across modalities. Models above the red dotted line are not significantly affected by the drift effect.}
    \label{fig:pval_vs_cos_pearson}
\end{figure}

The Pearson test measures a linear correlation between random variables, while the Spearman test measures a monotonic correlation. As there is no specific argument in favor of a linear relationship between the measured distributions (average cosine-similarity and norm of the average representation), we decided to use the Spearman correlation test in order to take into account more complex relation patterns.

Nevertheless, this metric is based on the rank of each observation, and is thus not robust to fluctuations due to sample variance, specifically when working with such small samples. This is reflected by the discrepancy between Pearson and Spearman p-values for some models (e.g. GPT-2).

\subsection{Cosine-similarity and anisotropy}
\begin{figure}[ht]
    \centering
    \includegraphics[width=\linewidth]{sources/part_1/anisotropy/imgs/cosine_v_density.png}
    \caption{Density function of cosine-similarity for a normal distribution as the dimension increases.}
    \label{fig:cosine_v_density}
\end{figure}
\begin{figure}[ht]
    \centering
    \includegraphics[width=\linewidth]{sources/part_1/anisotropy/imgs/q95_dimension.png}
    \caption{95th quartile of the cosine-similarity distribution on a normal distribution as the dimension increases. We add points for the average cosine-similarity level of Transformers models for several modalities.}
    \label{fig:q95}
\end{figure}

It can be argued that describing anisotropy as the observation of "high" cosine-similarity values is not a convincing definition. This section aims at showing which ranges of cosine-similarity values are characteristic of anisotropic distributions. 
In \autoref{fig:cosine_v_density}, we show the density function of the cosine-similarity values obtained when drawing pairs of samples from isotropic normal distributions in $\mathbb{R}^d$ as $d$ increases. 

For smaller dimensions ($d=16$), we see that the range of cosine-similarity values that are attained between isotropic distributions is relatively broad compared to the possible spectrum ($[-1, 1]$). As $d$ increases, the support of the observed distributions seems to become smaller, due to the curse of dimensionality.

We analyze this effect more in-depth in \autoref{fig:q95}, where we plot the 95th quantile of the cosine-similarity distribution in the isotropic scenario. We also add values for the layer-wise average cosine-similarity levels of typical models in several modalities for comparison. We can clearly observe that the levels of cosine-similarity observed in the representations of Transformers-based models are significantly unlikely to be observed in between samples drawn in isotropic normal distributions.

Nevertheless, as we go towards higher dimensional spaces for bigger models (e.g. Llama-65B from \citet{touvron2023llama} has 8192 hidden dimensions), we believe that it may be relevant to introduce isotropy metrics that are grounded to isotropic cosine-similarity distributions. We leave this question for future works.

\subsection{Other projections for $Q_s$ and $K_s$}
\label{sec:other_projs}
As mentioned in the Discussion (\autoref{sec:discussion}), we reproduce visualizations from \autoref{sec:qk} using different projection choices. Namely, we compute the SVD on $K_s$ only in \autoref{fig:proj_qk_heads_K} and \autoref{fig:QK_dir_K}, and on $Q_s$ only in \autoref{fig:proj_qk_heads_Q} and \autoref{fig:QK_dir_Q}.

The plots show that not only does the distribution used for the SVD drifts away from the origin along training, but also that the other distribution drifts away from the origin in an opposite direction. In other words, the singular components of each distribution are also relevant to describe the drift of the other distribution. Hence, \autoref{fig:proj_qk_heads_K} and \autoref{fig:proj_qk_heads_Q} support our conclusion that the drift directions of keys and queries are aligned during training.


\begin{figure*}[ht]
    \centering
    \begin{subfigure}[b]{0.24\linewidth}
         \includegraphics[width=\linewidth]{sources/part_1/anisotropy/imgs/dist_l9h9_s0_K.png}
         \caption{Step 0}
         \label{fig:dist_qk_s0_K}
    \end{subfigure}
    \begin{subfigure}[b]{0.24\linewidth}
         \includegraphics[width=\linewidth]{sources/part_1/anisotropy/imgs/dist_l9h9_s40_K.png}
         \caption{Step 40k}
         \label{fig:dist_qk_s40_K}
    \end{subfigure}
    \begin{subfigure}[b]{0.24\linewidth}
         \includegraphics[width=\linewidth]{sources/part_1/anisotropy/imgs/dist_l9h9_s200_K.png}
         \caption{Step 200k}
         \label{fig:dist_qk_s200_K}
    \end{subfigure}
    \begin{subfigure}[b]{0.24\linewidth}
         \includegraphics[width=\linewidth]{sources/part_1/anisotropy/imgs/dist_l9h9_s2000_K.png}
         \caption{Step 2M (final)}
         \label{fig:dist_qk_s2M_K}
    \end{subfigure}
    \caption{Evolution of $Q_s$ and $K_s$ distributions along training. Vectors are projected using the SVD computed on $K_s$.}
    \label{fig:proj_qk_heads_K}
\end{figure*}

\begin{figure*}[ht]
    \centering
    \begin{subfigure}[b]{0.24\linewidth}
         \includegraphics[width=\linewidth]{sources/part_1/anisotropy/imgs/dist_l9h9_s0_Q.png}
         \caption{Step 0}
         \label{fig:dist_qk_s0_Q}
    \end{subfigure}
    \begin{subfigure}[b]{0.24\linewidth}
         \includegraphics[width=\linewidth]{sources/part_1/anisotropy/imgs/dist_l9h9_s40_Q.png}
         \caption{Step 40k}
         \label{fig:dist_qk_s40_Q}
    \end{subfigure}
    \begin{subfigure}[b]{0.24\linewidth}
         \includegraphics[width=\linewidth]{sources/part_1/anisotropy/imgs/dist_l9h9_s200_Q.png}
         \caption{Step 200k}
         \label{fig:dist_qk_s200_Q}
    \end{subfigure}
    \begin{subfigure}[b]{0.24\linewidth}
         \includegraphics[width=\linewidth]{sources/part_1/anisotropy/imgs/dist_l9h9_s2000_Q.png}
         \caption{Step 2M (final)}
         \label{fig:dist_qk_s2M_Q}
    \end{subfigure}
    \caption{Evolution of $Q_s$ and $K_s$ distributions along training. Vectors are projected using the SVD computed on $Q_s$.}
    \label{fig:proj_qk_heads_Q}
\end{figure*}

\begin{figure}[ht!]
    \centering
    \begin{subfigure}[b]{0.48\columnwidth}
         \includegraphics[width=\linewidth]{sources/part_1/anisotropy/imgs/l0h3_samedir_QK_K.png}
         \caption{Similar}
         \label{fig:QK_simdir_K}
    \end{subfigure}
    \begin{subfigure}[b]{0.48\columnwidth}
         \includegraphics[width=\linewidth]{sources/part_1/anisotropy/imgs/l9h5_diffdir_QK_K.png}
         \caption{Opposite}
         \label{fig:QK_diffdir_K}
    \end{subfigure}
    \caption{Evolution of $\bar{Q_s}$ and $\bar{K_s}$ along training for two different heads in the network, projected via the SVD of $K_s$.pip3 install ninja einops packaging
    }
    \label{fig:QK_dir_K}
\end{figure}

\begin{figure}[ht!]
    \centering
    \begin{subfigure}[b]{0.48\columnwidth}
         \includegraphics[width=\linewidth]{sources/part_1/anisotropy/imgs/l0h3_samedir_QK_Q.png}
         \caption{Similar}
         \label{fig:QK_simdir_Q}
    \end{subfigure}
    \begin{subfigure}[b]{0.48\columnwidth}
         \includegraphics[width=\linewidth]{sources/part_1/anisotropy/imgs/l9h5_diffdir_QK_Q.png}
         \caption{Opposite}
         \label{fig:QK_diffdir_Q}
    \end{subfigure}
    \caption{Evolution of $\bar{Q_s}$ and $\bar{K_s}$ along training for two different heads in the network, projected via the SVD of $Q_s$.}
    \label{fig:QK_dir_Q}
\end{figure}

\subsection{Stability across MultiBERT seeds}
\begin{figure*}[ht]
    \centering
    \begin{subfigure}[b]{0.24\linewidth}
         \includegraphics[width=\linewidth]{sources/part_1/anisotropy/imgs/seeds_qk_l0.png}
         \caption{Layer 0}
         \label{fig:seeds_l0}
    \end{subfigure}
    \begin{subfigure}[b]{0.24\linewidth}
         \includegraphics[width=\linewidth]{sources/part_1/anisotropy/imgs/seeds_qk_l2.png}
         \caption{Layer 2}
         \label{fig:seeds_l2}
    \end{subfigure}
    \begin{subfigure}[b]{0.24\linewidth}
         \includegraphics[width=\linewidth]{sources/part_1/anisotropy/imgs/seeds_qk_l6.png}
         \caption{Layer 6}
         \label{fig:seeds_l6}
    \end{subfigure}
    \begin{subfigure}[b]{0.24\linewidth}
         \includegraphics[width=\linewidth]{sources/part_1/anisotropy/imgs/seeds_qk_l11.png}
         \caption{Layer 11}
         \label{fig:seeds_l11}
    \end{subfigure}
    \caption{Evolution of cosine-similarity between $\bar{Q_s}$ and $\bar{K_s}$ along training for various initialization seeds. Representations are concatenated across heads, and each color represents one seed of the MultiBERT models. We observe similar trends across seeds.}
    \label{fig:seeds_qk}
\end{figure*}


\label{chap:anisotropy}

\chapter*{Conclusion}

Overall, studying distortions and biases in the representation space has allowed us to shed light on bottlenecks and limitations that are inherent to the classical language modeling framework. It also provided insights about the architecture of modern language models, from the dimensionality and sparsity perspectives. 

Beyond the scope of usual interpretability frameworks, that are designed to explain predictions from targeted observations, we advocate for tools that allow analyzing global behaviors of the inner states of language models, in order to provide guidelines towards better paradigm for learning models of natural language.

Finally, we underline that our work, especially \Cref{chap:geobias} and \Cref{chap:softmax_bottleneck}, shows that representation degeneration and frequency-related biases hurt the quality of affected language models, either by degrading their performance or incorporating knowledge bias. In \Cref{part:solutions}, we propose several methods aimed at avoiding degeneration, reducing frequency dependency and mitigating sparsity in language models, in the hope that these methods will indirectly mitigate the identified limitations that are correlated with these phenomena.

% %%%%%%%%%%%%%%%%%%%%%%%%%%%%%%%%%%%%%%%%%%%%%%%%%%%%%%%%%%%%%%%%%%%%%%%%
\section{Representation Learning}
%%%%%%%%%%%%%%%%%%%%%%%%%%%%%%%%%%%%%%%%%%%%%%%%%%%%%%%%%%%%%%%%%%%%%%%%


\subsection{Introduction}

There are many different ways to represent textual data informatically. Text can be stored as bytes that encode written symbols, but it can also be read orally and recorded into a sound file, or stored in a numerical image as part of a screenshot. Hence, when designing algorithms that process natural language, one should pay attention to the nature of the \textit{features} that represent a given utterance in order to optimize for performance and efficiency.

Usually, the \textit{representation} of an object is a real-valued vector which makes it \textit{easier to extract useful information when building classifiers and other predictors} \citep{bengio_repr}. In the case of Natural Language Processing, the represented objects can be of various types and granularities:
\begin{itemize}
  \item \textbf{Language} : initiatives such as the World Atlas of Language Structures \citep{wals} have 
\end{itemize}




%%%%%%%%%%%%%%%%%%%%%%%%%%%%%%%%%%%%%%%%%%%%%%%%%%%%%%%%%%%%%%%%%%%%%%%%
\subsection{Statistical approaches}
%%%%%%%%%%%%%%%%%%%%%%%%%%%%%%%%%%%%%%%%%%%%%%%%%%%%%%%%%%%%%%%%%%%%%%%%
Statistical approaches to representation learning primarily involve methods that leverage co-occurrence statistics and distributional properties of words. Key techniques in this category include:

\begin{itemize}
  \item Latent Semantic Analysis (LSA): LSA is based on the Singular Value Decomposition (SVD) of term-document matrices, reducing the dimensionality of the data and uncovering latent semantic structures. By mapping words and documents to a shared vector space, LSA captures semantic similarities based on co-occurrence patterns.
\end{itemize}


Latent Dirichlet Allocation (LDA): LDA is a generative probabilistic model that represents documents as mixtures of topics, where each topic is a distribution over words. By inferring the topic distribution for each document, LDA provides a way to represent documents in a lower-dimensional topic space.

Word2Vec: Introduced by Mikolov et al., Word2Vec includes two model architectures—Continuous Bag of Words (CBOW) and Skip-gram. These models learn word embeddings by predicting the context words surrounding a target word or vice versa. The resulting vectors capture semantic relationships such as analogies (e.g., "king" - "man" + "woman" $\simeq$ "queen").

GloVe (Global Vectors for Word Representation): GloVe is another word embedding technique that combines the advantages of matrix factorization and local context window methods. It constructs a word-word co-occurrence matrix and derives word vectors by factorizing this matrix, ensuring that the dot product of word vectors approximates the logarithm of their co-occurrence probabilities.

%%%%%%%%%%%%%%%%%%%%%%%%%%%%%%%%%%%%%%%%%%%%%%%%%%%%%%%%%%%%%%%%%%%%%%%%
\subsection{Auto-encoders}
%%%%%%%%%%%%%%%%%%%%%%%%%%%%%%%%%%%%%%%%%%%%%%%%%%%%%%%%%%%%%%%%%%%%%%%%
Auto-encoders are neural network models designed to learn efficient representations of data through unsupervised learning. They consist of an encoder that maps input data to a latent space and a decoder that reconstructs the original data from this latent representation. In NLP, auto-encoders can be used to learn embeddings for words, sentences, or documents.

Basic Auto-Encoders: The simplest form of auto-encoders involves a single hidden layer that compresses the input into a lower-dimensional latent space. The model is trained to minimize the reconstruction error between the input and the output.

Variational Auto-Encoders (VAEs): VAEs extend basic auto-encoders by imposing a probabilistic structure on the latent space. They use a probabilistic encoder to map inputs to a distribution in the latent space, allowing for the generation of new samples by sampling from this distribution. VAEs are useful in tasks requiring generative capabilities, such as text generation.

Denoising Auto-Encoders (DAEs): DAEs are trained to reconstruct the original data from corrupted versions. This process encourages the model to learn robust features that are invariant to noise, improving the quality of the learned representations.

\subsection{Contrastive approaches}

Contrastive approaches in representation learning aim to learn effective embeddings by contrasting positive and negative examples. The core idea is to bring similar items closer together in the embedding space while pushing dissimilar items apart. These methods are essential for capturing nuanced relationships in the data and enhancing the quality of learned representations.
\begin{itemize}
  \item Contrastive Loss: The fundamental concept in contrastive learning is the contrastive loss function, which drives the learning process by encouraging the model to distinguish between positive pairs (similar items) and negative pairs (dissimilar items).
  \item Triplet Loss: Triplet loss is a popular contrastive learning technique that uses triplets of samples: an anchor, a positive (similar to the anchor), and a negative (dissimilar to the anchor). The objective is to minimize the distance between the anchor and the positive while maximizing the distance between the anchor and the negative. This approach is widely used in tasks such as face recognition and text similarity.
  \item 
  Noise Contrastive Estimation (NCE): NCE is another contrastive learning method that reformulates the problem of estimating a probability distribution into a binary classification problem. The model learns to distinguish between observed data and artificially generated noise samples. NCE is particularly useful in large-scale language models where direct computation of probabilities is computationally expensive.
\end{itemize}
% %%%%%%%%%%%%%%%%%%%%%%%%%%%%%%%%%%%%%%%%%%%%%%%%%%%%%%%%%%%%%%%%%%%%%%%%
\section{Language Modeling}
%%%%%%%%%%%%%%%%%%%%%%%%%%%%%%%%%%%%%%%%%%%%%%%%%%%%%%%%%%%%%%%%%%%%%%%%


%%%%%%%%%%%%%%%%%%%%%%%%%%%%%%%%%%%%%%%%%%%%%%%%%%%%%%%%%%%%%%%%%%%%%%%%
\subsection{Introduction}
%%%%%%%%%%%%%%%%%%%%%%%%%%%%%%%%%%%%%%%%%%%%%%%%%%%%%%%%%%%%%%%%%%%%%%%%

Language modeling is a fundamental task in natural language processing (NLP) that involves predicting the next word in a sequence given the preceding context. This task is crucial for various applications, including speech recognition, machine translation, text generation, and more. Language models capture the probability distribution of word sequences, enabling them to generate coherent and contextually appropriate text.

Historically, language models relied on n-gram approaches, which predict a word based on the previous n-1 words. However, these models faced limitations in handling long-range dependencies and sparsity issues. With the advent of deep learning, neural language models have significantly advanced the field, providing more powerful and flexible approaches to capturing language patterns.

%%%%%%%%%%%%%%%%%%%%%%%%%%%%%%%%%%%%%%%%%%%%%%%%%%%%%%%%%%%%%%%%%%%%%%%%
\subsection{Methods}

The process of training a language model involves several key steps, from preparing the text data to optimizing the model using specific objectives. Here is a simplified pipeline:

\begin{itemize}
    \item \textbf{Tokenization}: The first step in building a language model is tokenization, which involves breaking down the text into smaller units called tokens. Tokens can be words, subwords, or characters. This step converts raw text into a format that the model can process, typically resulting in a sequence of token indices.

    \item \textbf{Embedding}: Once the text is tokenized, each token is mapped to a continuous vector representation, known as an embedding. These embeddings capture semantic information about the tokens and are learned during training.

    \item \textbf{Contextualization}: The model processes the sequence of token embeddings to capture the contextual relationships between tokens. This involves passing the embeddings through layers of neural networks that refine the representations based on the surrounding context.

    \item \textbf{Prediction}: For each position in the sequence, the model predicts the probability distribution of the next token. This step involves transforming the contextualized representations into logits, which are unnormalized scores for each token in the vocabulary.

    \item \textbf{Objective Function}: The model is trained to minimize a specific objective function that measures the difference between the predicted probabilities and the actual next token. The most common objective function for language modeling is cross-entropy loss, which quantifies the accuracy of the model's predictions.

    \item \textbf{Contrastive Methods}: In addition to traditional cross-entropy loss, contrastive methods can be used to improve the quality of the learned representations. These methods involve creating pairs of similar and dissimilar examples and training the model to distinguish between them, enhancing the model's ability to capture nuanced relationships in the data.

    \item \textbf{Regularization}: To prevent overfitting and improve generalization, regularization techniques are applied during training. Common regularization methods include dropout (randomly dropping units during training to prevent co-adaptation), weight decay (penalizing large weights), and data augmentation (creating variations of the training data to improve robustness).
\end{itemize}

In summary, the objective pipeline for training a language model involves tokenizing the text, embedding the tokens, capturing contextual relationships, making predictions, and optimizing the model using objective functions like cross-entropy and contrastive loss, along with regularization techniques to ensure robust and effective learning.


\subsection{Architectures}

The architecture of a language model significantly influences its performance and capabilities. Key architectures include Recurrent Neural Networks (RNNs) and Transformers, each with unique mechanisms for processing sequences.

\begin{itemize}
    \item \textbf{Recurrent Neural Networks (RNNs)}: RNNs process sequences one element at a time, maintaining a hidden state that captures information from previous steps. Variants like Long Short-Term Memory (LSTM) and Gated Recurrent Units (GRUs) include gating mechanisms to mitigate issues like vanishing and exploding gradients. These gates control the flow of information, allowing the network to capture long-range dependencies more effectively.

    \item \textbf{Transformers}: Transformers use self-attention mechanisms to weigh the importance of different words in a sequence relative to each other. They are composed of an encoder and a decoder:

    \begin{itemize}
        \item \textbf{Encoder}: The encoder consists of multiple layers, each containing two main components: a multi-head self-attention mechanism and a position-wise feedforward neural network. The self-attention mechanism allows the model to consider all positions in the input sequence simultaneously, capturing dependencies regardless of their distance.
        
        \item \textbf{Decoder}: The decoder also consists of multiple layers, with three main components: a masked multi-head self-attention mechanism, an encoder-decoder attention mechanism, and a position-wise feedforward neural network. The masked self-attention ensures that each position can only attend to earlier positions, maintaining the autoregressive property during generation.
    \end{itemize}

    \item \textbf{Attention Mechanisms}: The attention mechanisms in transformers come in two forms:
    \begin{itemize}
        \item \textbf{Masked Attention}: Used in the decoder, masked attention ensures that the model cannot attend to future tokens, preserving the causality needed for autoregressive tasks like text generation.
        
        \item \textbf{Causal Attention}: Similar to masked attention, causal attention restricts the attention to past and present tokens only, which is essential for maintaining the correct sequence of predictions.
        
        \item \textbf{Self-Attention}: Used in both the encoder and decoder, self-attention allows each token to attend to all other tokens in the sequence, capturing global dependencies and contextual information.
    \end{itemize}

    \item \textbf{Bidirectional vs. Unidirectional Models}:
    \begin{itemize}
        \item \textbf{Bidirectional Models}: Examples include BERT (Bidirectional Encoder Representations from Transformers), which attends to both past and future contexts in the input sequence. This is useful for tasks requiring comprehensive context understanding, such as question answering and sentiment analysis.
        
        \item \textbf{Unidirectional Models}: Examples include GPT (Generative Pre-trained Transformer), which attends only to past tokens. This autoregressive approach is particularly effective for text generation tasks.
    \end{itemize}
    
    \item \textbf{Encoder-Decoder Models}: Models like T5 (Text-to-Text Transfer Transformer) utilize both encoder and decoder structures. The encoder processes the input sequence into a context-rich representation, which the decoder then uses to generate the output sequence. This architecture is versatile, handling a wide range of text-to-text tasks under a unified framework.
\end{itemize}

In summary, the architecture of language models ranges from RNNs that process sequences sequentially to transformers that leverage self-attention for capturing dependencies across entire sequences. These architectures, with their various attention mechanisms and structural differences, enable powerful and flexible modeling of natural language.


%%%%%%%%%%%%%%%%%%%%%%%%%%%%%%%%%%%%%%%%%%%%%%%%%%%%%%%%%%%%%%%%%%%%%%%%
\subsection{Limitations}

Despite their advancements, language models face several limitations:

\begin{itemize}
    \item \textbf{Data and Computation Requirements}: Training state-of-the-art language models requires vast amounts of data and computational resources. This limitation restricts access to such models to organizations with substantial resources and makes the training process energy-intensive.

    \item \textbf{Bias and Fairness}: Language models can learn and perpetuate biases present in the training data, leading to biased and potentially harmful outputs. Addressing these biases is a critical area of ongoing research, as it impacts the fairness and ethical use of NLP systems.

    \item \textbf{Context Length}: While transformers have improved the handling of long-range dependencies, they are still limited by the maximum input length they can process. Techniques like segment-level recurrence or hierarchical models are being explored to address this limitation.

    \item \textbf{Interpretability}: Deep neural language models are often seen as black boxes, making it challenging to understand and interpret their predictions. Enhancing the interpretability of these models is essential for building trust and ensuring their safe application.

    \item \textbf{Generalization}: Language models sometimes struggle to generalize to out-of-distribution examples or novel contexts not seen during training. Ensuring robust generalization remains an important challenge, particularly for applications in dynamic or unpredictable environments.
\end{itemize}

In summary, while language models have made significant strides in NLP, they face notable challenges related to data and computation requirements, bias and fairness, context length, interpretability, and generalization. Addressing these limitations is crucial for the continued advancement and ethical deployment of NLP technologies.


% %%%%%%%%%%%%%%%%%%%%%%%%%%%%%%%%%%%%%%%%%%%%%%%%%%%%%%%%%%%%%%%%%%%%%%%%
\chapter{Related Works}
%%%%%%%%%%%%%%%%%%%%%%%%%%%%%%%%%%%%%%%%%%%%%%%%%%%%%%%%%%%%%%%%%%%%%%%%

%%%%%%%%%%%%%%%%%%%%%%%%%%%%%%%%%%%%%%%%%%%%%%%%%%%%%%%%%%%%%%%%%%%%%%%%
%%%%%%%%%%%%%%%%%%%%%%%%%%%%%%%%%%%%%%%%%%%%%%%%%%%%%%%%%%%%%%%%%%%%%%%%
\section{Representation Learning}
%%%%%%%%%%%%%%%%%%%%%%%%%%%%%%%%%%%%%%%%%%%%%%%%%%%%%%%%%%%%%%%%%%%%%%%%


\subsection{Introduction}

There are many different ways to represent textual data informatically. Text can be stored as bytes that encode written symbols, but it can also be read orally and recorded into a sound file, or stored in a numerical image as part of a screenshot. Hence, when designing algorithms that process natural language, one should pay attention to the nature of the \textit{features} that represent a given utterance in order to optimize for performance and efficiency.

Usually, the \textit{representation} of an object is a real-valued vector which makes it \textit{easier to extract useful information when building classifiers and other predictors} \citep{bengio_repr}. In the case of Natural Language Processing, the represented objects can be of various types and granularities:
\begin{itemize}
  \item \textbf{Language} : initiatives such as the World Atlas of Language Structures \citep{wals} have 
\end{itemize}




%%%%%%%%%%%%%%%%%%%%%%%%%%%%%%%%%%%%%%%%%%%%%%%%%%%%%%%%%%%%%%%%%%%%%%%%
\subsection{Statistical approaches}
%%%%%%%%%%%%%%%%%%%%%%%%%%%%%%%%%%%%%%%%%%%%%%%%%%%%%%%%%%%%%%%%%%%%%%%%
Statistical approaches to representation learning primarily involve methods that leverage co-occurrence statistics and distributional properties of words. Key techniques in this category include:

\begin{itemize}
  \item Latent Semantic Analysis (LSA): LSA is based on the Singular Value Decomposition (SVD) of term-document matrices, reducing the dimensionality of the data and uncovering latent semantic structures. By mapping words and documents to a shared vector space, LSA captures semantic similarities based on co-occurrence patterns.
\end{itemize}


Latent Dirichlet Allocation (LDA): LDA is a generative probabilistic model that represents documents as mixtures of topics, where each topic is a distribution over words. By inferring the topic distribution for each document, LDA provides a way to represent documents in a lower-dimensional topic space.

Word2Vec: Introduced by Mikolov et al., Word2Vec includes two model architectures—Continuous Bag of Words (CBOW) and Skip-gram. These models learn word embeddings by predicting the context words surrounding a target word or vice versa. The resulting vectors capture semantic relationships such as analogies (e.g., "king" - "man" + "woman" $\simeq$ "queen").

GloVe (Global Vectors for Word Representation): GloVe is another word embedding technique that combines the advantages of matrix factorization and local context window methods. It constructs a word-word co-occurrence matrix and derives word vectors by factorizing this matrix, ensuring that the dot product of word vectors approximates the logarithm of their co-occurrence probabilities.

%%%%%%%%%%%%%%%%%%%%%%%%%%%%%%%%%%%%%%%%%%%%%%%%%%%%%%%%%%%%%%%%%%%%%%%%
\subsection{Auto-encoders}
%%%%%%%%%%%%%%%%%%%%%%%%%%%%%%%%%%%%%%%%%%%%%%%%%%%%%%%%%%%%%%%%%%%%%%%%
Auto-encoders are neural network models designed to learn efficient representations of data through unsupervised learning. They consist of an encoder that maps input data to a latent space and a decoder that reconstructs the original data from this latent representation. In NLP, auto-encoders can be used to learn embeddings for words, sentences, or documents.

Basic Auto-Encoders: The simplest form of auto-encoders involves a single hidden layer that compresses the input into a lower-dimensional latent space. The model is trained to minimize the reconstruction error between the input and the output.

Variational Auto-Encoders (VAEs): VAEs extend basic auto-encoders by imposing a probabilistic structure on the latent space. They use a probabilistic encoder to map inputs to a distribution in the latent space, allowing for the generation of new samples by sampling from this distribution. VAEs are useful in tasks requiring generative capabilities, such as text generation.

Denoising Auto-Encoders (DAEs): DAEs are trained to reconstruct the original data from corrupted versions. This process encourages the model to learn robust features that are invariant to noise, improving the quality of the learned representations.

\subsection{Contrastive approaches}

Contrastive approaches in representation learning aim to learn effective embeddings by contrasting positive and negative examples. The core idea is to bring similar items closer together in the embedding space while pushing dissimilar items apart. These methods are essential for capturing nuanced relationships in the data and enhancing the quality of learned representations.
\begin{itemize}
  \item Contrastive Loss: The fundamental concept in contrastive learning is the contrastive loss function, which drives the learning process by encouraging the model to distinguish between positive pairs (similar items) and negative pairs (dissimilar items).
  \item Triplet Loss: Triplet loss is a popular contrastive learning technique that uses triplets of samples: an anchor, a positive (similar to the anchor), and a negative (dissimilar to the anchor). The objective is to minimize the distance between the anchor and the positive while maximizing the distance between the anchor and the negative. This approach is widely used in tasks such as face recognition and text similarity.
  \item 
  Noise Contrastive Estimation (NCE): NCE is another contrastive learning method that reformulates the problem of estimating a probability distribution into a binary classification problem. The model learns to distinguish between observed data and artificially generated noise samples. NCE is particularly useful in large-scale language models where direct computation of probabilities is computationally expensive.
\end{itemize}

%%%%%%%%%%%%%%%%%%%%%%%%%%%%%%%%%%%%%%%%%%%%%%%%%%%%%%%%%%%%%%%%%%%%%%%%
\section{Language Modeling}
%%%%%%%%%%%%%%%%%%%%%%%%%%%%%%%%%%%%%%%%%%%%%%%%%%%%%%%%%%%%%%%%%%%%%%%%


%%%%%%%%%%%%%%%%%%%%%%%%%%%%%%%%%%%%%%%%%%%%%%%%%%%%%%%%%%%%%%%%%%%%%%%%
\subsection{Introduction}
%%%%%%%%%%%%%%%%%%%%%%%%%%%%%%%%%%%%%%%%%%%%%%%%%%%%%%%%%%%%%%%%%%%%%%%%

Language modeling is a fundamental task in natural language processing (NLP) that involves predicting the next word in a sequence given the preceding context. This task is crucial for various applications, including speech recognition, machine translation, text generation, and more. Language models capture the probability distribution of word sequences, enabling them to generate coherent and contextually appropriate text.

Historically, language models relied on n-gram approaches, which predict a word based on the previous n-1 words. However, these models faced limitations in handling long-range dependencies and sparsity issues. With the advent of deep learning, neural language models have significantly advanced the field, providing more powerful and flexible approaches to capturing language patterns.

%%%%%%%%%%%%%%%%%%%%%%%%%%%%%%%%%%%%%%%%%%%%%%%%%%%%%%%%%%%%%%%%%%%%%%%%
\subsection{Methods}

The process of training a language model involves several key steps, from preparing the text data to optimizing the model using specific objectives. Here is a simplified pipeline:

\begin{itemize}
    \item \textbf{Tokenization}: The first step in building a language model is tokenization, which involves breaking down the text into smaller units called tokens. Tokens can be words, subwords, or characters. This step converts raw text into a format that the model can process, typically resulting in a sequence of token indices.

    \item \textbf{Embedding}: Once the text is tokenized, each token is mapped to a continuous vector representation, known as an embedding. These embeddings capture semantic information about the tokens and are learned during training.

    \item \textbf{Contextualization}: The model processes the sequence of token embeddings to capture the contextual relationships between tokens. This involves passing the embeddings through layers of neural networks that refine the representations based on the surrounding context.

    \item \textbf{Prediction}: For each position in the sequence, the model predicts the probability distribution of the next token. This step involves transforming the contextualized representations into logits, which are unnormalized scores for each token in the vocabulary.

    \item \textbf{Objective Function}: The model is trained to minimize a specific objective function that measures the difference between the predicted probabilities and the actual next token. The most common objective function for language modeling is cross-entropy loss, which quantifies the accuracy of the model's predictions.

    \item \textbf{Contrastive Methods}: In addition to traditional cross-entropy loss, contrastive methods can be used to improve the quality of the learned representations. These methods involve creating pairs of similar and dissimilar examples and training the model to distinguish between them, enhancing the model's ability to capture nuanced relationships in the data.

    \item \textbf{Regularization}: To prevent overfitting and improve generalization, regularization techniques are applied during training. Common regularization methods include dropout (randomly dropping units during training to prevent co-adaptation), weight decay (penalizing large weights), and data augmentation (creating variations of the training data to improve robustness).
\end{itemize}

In summary, the objective pipeline for training a language model involves tokenizing the text, embedding the tokens, capturing contextual relationships, making predictions, and optimizing the model using objective functions like cross-entropy and contrastive loss, along with regularization techniques to ensure robust and effective learning.


\subsection{Architectures}

The architecture of a language model significantly influences its performance and capabilities. Key architectures include Recurrent Neural Networks (RNNs) and Transformers, each with unique mechanisms for processing sequences.

\begin{itemize}
    \item \textbf{Recurrent Neural Networks (RNNs)}: RNNs process sequences one element at a time, maintaining a hidden state that captures information from previous steps. Variants like Long Short-Term Memory (LSTM) and Gated Recurrent Units (GRUs) include gating mechanisms to mitigate issues like vanishing and exploding gradients. These gates control the flow of information, allowing the network to capture long-range dependencies more effectively.

    \item \textbf{Transformers}: Transformers use self-attention mechanisms to weigh the importance of different words in a sequence relative to each other. They are composed of an encoder and a decoder:

    \begin{itemize}
        \item \textbf{Encoder}: The encoder consists of multiple layers, each containing two main components: a multi-head self-attention mechanism and a position-wise feedforward neural network. The self-attention mechanism allows the model to consider all positions in the input sequence simultaneously, capturing dependencies regardless of their distance.
        
        \item \textbf{Decoder}: The decoder also consists of multiple layers, with three main components: a masked multi-head self-attention mechanism, an encoder-decoder attention mechanism, and a position-wise feedforward neural network. The masked self-attention ensures that each position can only attend to earlier positions, maintaining the autoregressive property during generation.
    \end{itemize}

    \item \textbf{Attention Mechanisms}: The attention mechanisms in transformers come in two forms:
    \begin{itemize}
        \item \textbf{Masked Attention}: Used in the decoder, masked attention ensures that the model cannot attend to future tokens, preserving the causality needed for autoregressive tasks like text generation.
        
        \item \textbf{Causal Attention}: Similar to masked attention, causal attention restricts the attention to past and present tokens only, which is essential for maintaining the correct sequence of predictions.
        
        \item \textbf{Self-Attention}: Used in both the encoder and decoder, self-attention allows each token to attend to all other tokens in the sequence, capturing global dependencies and contextual information.
    \end{itemize}

    \item \textbf{Bidirectional vs. Unidirectional Models}:
    \begin{itemize}
        \item \textbf{Bidirectional Models}: Examples include BERT (Bidirectional Encoder Representations from Transformers), which attends to both past and future contexts in the input sequence. This is useful for tasks requiring comprehensive context understanding, such as question answering and sentiment analysis.
        
        \item \textbf{Unidirectional Models}: Examples include GPT (Generative Pre-trained Transformer), which attends only to past tokens. This autoregressive approach is particularly effective for text generation tasks.
    \end{itemize}
    
    \item \textbf{Encoder-Decoder Models}: Models like T5 (Text-to-Text Transfer Transformer) utilize both encoder and decoder structures. The encoder processes the input sequence into a context-rich representation, which the decoder then uses to generate the output sequence. This architecture is versatile, handling a wide range of text-to-text tasks under a unified framework.
\end{itemize}

In summary, the architecture of language models ranges from RNNs that process sequences sequentially to transformers that leverage self-attention for capturing dependencies across entire sequences. These architectures, with their various attention mechanisms and structural differences, enable powerful and flexible modeling of natural language.


%%%%%%%%%%%%%%%%%%%%%%%%%%%%%%%%%%%%%%%%%%%%%%%%%%%%%%%%%%%%%%%%%%%%%%%%
\subsection{Limitations}

Despite their advancements, language models face several limitations:

\begin{itemize}
    \item \textbf{Data and Computation Requirements}: Training state-of-the-art language models requires vast amounts of data and computational resources. This limitation restricts access to such models to organizations with substantial resources and makes the training process energy-intensive.

    \item \textbf{Bias and Fairness}: Language models can learn and perpetuate biases present in the training data, leading to biased and potentially harmful outputs. Addressing these biases is a critical area of ongoing research, as it impacts the fairness and ethical use of NLP systems.

    \item \textbf{Context Length}: While transformers have improved the handling of long-range dependencies, they are still limited by the maximum input length they can process. Techniques like segment-level recurrence or hierarchical models are being explored to address this limitation.

    \item \textbf{Interpretability}: Deep neural language models are often seen as black boxes, making it challenging to understand and interpret their predictions. Enhancing the interpretability of these models is essential for building trust and ensuring their safe application.

    \item \textbf{Generalization}: Language models sometimes struggle to generalize to out-of-distribution examples or novel contexts not seen during training. Ensuring robust generalization remains an important challenge, particularly for applications in dynamic or unpredictable environments.
\end{itemize}

In summary, while language models have made significant strides in NLP, they face notable challenges related to data and computation requirements, bias and fairness, context length, interpretability, and generalization. Addressing these limitations is crucial for the continued advancement and ethical deployment of NLP technologies.



%%%%%%%%%%%%%%%%%%%%%%%%%%%%%%%%%%%%%%%%%%%%%%%%%%%%%%%%%%%%%%%%%%%%%%%%
\section{Representation Analysis for NLP}
%%%%%%%%%%%%%%%%%%%%%%%%%%%%%%%%%%%%%%%%%%%%%%%%%%%%%%%%%%%%%%%%%%%%%%%%

\subsection{Representations and Linguistic Properties}

Representations learned by NLP models capture various linguistic properties that are essential for understanding language. These properties can be broadly categorized into syntactic and semantic information.

\begin{itemize}
    \item \textbf{Syntactic Properties}: Syntactic properties refer to the structural aspects of language, including grammar and sentence structure. Effective representations encode the following syntactic information:
    \begin{itemize}
        \item \textbf{Part-of-Speech Tags}: Representations can capture the grammatical categories of words, such as nouns, verbs, adjectives, etc. This helps in understanding the roles that words play in sentences.
        \item \textbf{Dependency Relations}: Words in a sentence often have grammatical dependencies with each other (e.g., subject-verb, adjective-noun). Representations that capture these dependencies can help in tasks like parsing and syntax-based translation.
        \item \textbf{Constituent Structure}: Representations may also encode higher-level syntactic structures, such as phrases and clauses, which are important for understanding the hierarchical organization of sentences.
        \item \textbf{Word Order}: The sequence in which words appear in a sentence is crucial for meaning. Representations that preserve word order information can help in tasks like machine translation and text generation.
    \end{itemize}

    \item \textbf{Semantic Properties}: Semantic properties involve the meanings of words and their relationships. Effective representations capture the following semantic information:
    \begin{itemize}
        \item \textbf{Word Meanings}: Representations encode the meanings of individual words. This can be analyzed through tasks like word similarity, where similar words (e.g., "cat" and "feline") have similar embeddings.
        \item \textbf{Contextual Meaning}: The meaning of a word can change based on its context. Contextual embeddings, such as those from models like BERT, capture these nuances by considering surrounding words. For example, the word "bank" has different meanings in "river bank" and "financial bank".
        \item \textbf{Synonymy and Antonymy}: Effective representations capture semantic relationships such as synonyms (words with similar meanings) and antonyms (words with opposite meanings). This is crucial for tasks like paraphrase detection and sentiment analysis.
        \item \textbf{Polysemy}: Words with multiple meanings (polysemous words) should have representations that reflect their different senses depending on context. For instance, "bark" should have different embeddings when referring to a tree's outer layer versus a dog's sound.
        \item \textbf{Compositionality}: The meaning of phrases and sentences is often compositional, meaning it is derived from the meanings of individual words and their arrangement. Representations that capture compositionality help in understanding complex expressions and idiomatic phrases.
    \end{itemize}
\end{itemize}

To analyze these properties, various probing techniques are used:
\begin{itemize}
    \item \textbf{Probing Classifiers}: Small supervised classifiers are trained on top of fixed embeddings to predict linguistic properties like part-of-speech tags, syntactic roles, or semantic roles. High accuracy indicates that the embeddings capture relevant linguistic information.
    \item \textbf{Visualization}: Techniques such as t-SNE or PCA can visualize the high-dimensional embeddings in a lower-dimensional space, helping to inspect clusters and relationships between words.
    \item \textbf{Correlation Analysis}: Correlating embedding distances with human-judged linguistic distances (e.g., similarity or relatedness scores) can provide insights into how well the representations capture semantic relationships.
    \item \textbf{Linguistic Tasks}: Evaluating representations on downstream linguistic tasks, such as named entity recognition, sentiment analysis, or syntactic parsing, provides practical evidence of the embeddings' effectiveness in capturing linguistic properties.
\end{itemize}

In summary, representations learned by NLP models encapsulate both syntactic and semantic properties of language. Analyzing these representations through various probing techniques helps in understanding their effectiveness and guiding further improvements in model design and training.


\subsection{Analyzing Self-Attention}

Self-attention mechanisms in transformer models allow for the examination of how tokens attend to each other, providing insights into the model's internal workings. Analyzing self-attention helps to understand what information the model considers important and how it processes different parts of the input sequence.

\begin{itemize}
    \item \textbf{Attention Patterns}: By visualizing attention weights, we can analyze how the model distributes attention across different tokens. Attention patterns reveal which tokens are considered relevant for predicting the next token in a sequence. Typical visualization techniques include attention heatmaps and attention heads visualizations. These patterns can show whether the model focuses on nearby words, distant words, or specific syntactic structures.
    
    \item \textbf{Head Specialization}: Transformers use multi-head self-attention mechanisms, where multiple attention heads operate in parallel. Each head can learn to focus on different types of relationships or aspects of the input. Analyzing head specialization involves examining the distinct roles of each attention head. For example, some heads might specialize in capturing syntactic dependencies (like subject-verb relationships), while others might focus on semantic roles (like identifying entities and their attributes).

    \item \textbf{Layer-wise Analysis}: Self-attention can be analyzed at different layers of the transformer. Lower layers often capture more local and syntactic information, while higher layers tend to capture more global and semantic information. Layer-wise analysis helps in understanding the hierarchical nature of learned representations and how information is progressively abstracted.
    
    \item \textbf{Global vs. Local Attention}: Analyzing whether the model's attention is more global (considering distant tokens) or local (focusing on nearby tokens) helps in understanding its contextual understanding. For instance, attention to distant tokens can indicate the model's ability to capture long-range dependencies.

    \item \textbf{Attention as Explanation}: Attention weights are sometimes used as explanations for model predictions. However, it is important to note that while attention provides some interpretability, it is not a definitive explanation of model behavior. Additional analysis and methods are often needed to fully understand the model's decision-making process.
    
\end{itemize}

In summary, analyzing self-attention mechanisms in transformer models provides valuable insights into how these models process and prioritize different parts of the input sequence. Visualization and interpretation of attention patterns, head specialization, and layer-wise behavior help in understanding the internal workings of the model and improving its performance.


\subsection{Similarity and Geometry}

The geometric properties of the learned representations provide valuable insights into the structure and effectiveness of the embedding space. Understanding similarity and geometry is crucial for evaluating how well the model captures relationships between words and phrases.

\begin{itemize}
    \item \textbf{Similarity Metrics}: Analyzing similarity metrics helps in understanding how close or distant different word embeddings are within the vector space.
    \begin{itemize}
        \item \textbf{Cosine Similarity}: This metric measures the cosine of the angle between two vectors, indicating how similar they are in terms of direction. High cosine similarity between embeddings suggests that the words are semantically similar.
        \item \textbf{Euclidean Distance}: This metric measures the straight-line distance between two points in the embedding space. Smaller distances indicate greater similarity. While less commonly used than cosine similarity, it can provide additional insights into the embedding space's structure.
    \end{itemize}
    
    \item \textbf{Clustering}: Grouping similar word embeddings together can reveal natural clusters within the embedding space.
    \begin{itemize}
        \item \textbf{K-Means Clustering}: This algorithm partitions the embedding space into \( k \) clusters, where each word belongs to the cluster with the nearest mean. This can reveal semantic groupings, such as synonyms or related concepts.
        \item \textbf{Hierarchical Clustering}: This method builds a hierarchy of clusters, which can be visualized as a dendrogram. It provides a more detailed view of the relationships between embeddings at different levels of granularity.
    \end{itemize}
    
    \item \textbf{Dimensionality Reduction}: Visualizing high-dimensional embeddings in a lower-dimensional space can help in understanding their geometric properties.
    \begin{itemize}
        \item \textbf{t-SNE (t-Distributed Stochastic Neighbor Embedding)}: This technique reduces the dimensionality of embeddings while preserving local structures, making it useful for visualizing clusters and relationships.
        \item \textbf{PCA (Principal Component Analysis)}: PCA reduces the dimensionality by projecting the embeddings onto the directions of maximum variance. This helps in identifying the principal components that capture most of the variance in the data.
    \end{itemize}

    \item \textbf{Embedding Space Geometry}: Studying the geometric properties of the embedding space provides insights into how well the model organizes linguistic information.
    \begin{itemize}
        \item \textbf{Density and Distribution}: Analyzing the density and distribution of embeddings can reveal whether the space is uniformly populated or contains sparse regions. A well-distributed space indicates good coverage of the language.
        \item \textbf{Subspace Structures}: Identifying subspaces within the embedding space that correspond to specific linguistic features (e.g., tense, number, or gender) can provide insights into how these features are encoded. For example, certain directions in the embedding space may correspond to semantic shifts like singular to plural forms.
    \end{itemize}

    \item \textbf{Analogies and Linear Relationships}: Embeddings often capture analogical relationships through linear transformations. For instance, the relationship between "king" and "queen" can be similar to the relationship between "man" and "woman."
    \begin{itemize}
        \item \textbf{Word Analogies}: By performing vector arithmetic (e.g., "king" - "man" + "woman"), one can retrieve vectors close to the expected answer (e.g., "queen"). This demonstrates the model's ability to capture meaningful relationships.
        \item \textbf{Linear Projections}: Identifying and interpreting linear projections that correspond to specific semantic or syntactic properties can help in understanding the embedding space's structure. For example, projecting embeddings onto the gender subspace can reveal gender biases.
    \end{itemize}
    
    \item \textbf{Intrinsic Evaluation Tasks}: These tasks evaluate the quality of word embeddings based on their geometric properties.
    \begin{itemize}
        \item \textbf{Word Similarity Tasks}: These tasks measure how well the similarity between word embeddings aligns with human-judged similarities. Common datasets include WordSim-353 and SimLex-999.
        \item \textbf{Word Analogies Tasks}: These tasks evaluate the model's ability to solve analogy problems, such as "man is to king as woman is to ?". The accuracy in these tasks reflects the model's capability to capture linear relationships.
    \end{itemize}
    
\end{itemize}

In summary, analyzing the similarity and geometry of learned representations involves examining similarity metrics, clustering, dimensionality reduction, and intrinsic evaluation tasks. These analyses provide insights into the structure of the embedding space and the quality of the captured linguistic relationships.


\subsection{Representation Degeneration}

Representation degeneration refers to geometric issues in learned embeddings where the embeddings lose their discriminative power and become less effective at capturing meaningful distinctions. This can manifest in several geometric phenomena:

\begin{itemize}
    \item \textbf{Anisotropy}: Anisotropy in the embedding space occurs when the space is stretched or distorted in specific directions, causing the representations to be unevenly distributed. In the context of language models, anisotropic spaces can result in embeddings that are more spread out in certain dimensions while being compressed in others. This can affect the model’s ability to capture and differentiate between various semantic and syntactic features. 

    \item \textbf{Outlier Dimensions}: Outlier dimensions are directions in the embedding space that do not capture meaningful information and may represent noise or irrelevant features. These dimensions can distort the embeddings, leading to poor performance on tasks that rely on accurate semantic and syntactic understanding. Identifying and addressing outlier dimensions is essential for improving the quality of embeddings.

    \item \textbf{Representation Collapse}: Representation collapse refers to the phenomenon where embeddings of different tokens become indistinguishable and collapse into a narrow subspace. This often occurs when embeddings lose their diversity and become too similar to each other. Representation collapse reduces the model's ability to differentiate between tokens, adversely affecting downstream task performance. It can be detected by analyzing the clustering of embeddings or by examining their distribution.

    \item \textbf{Biases in Latent Spaces}: Embedding spaces can encode biases present in the training data, leading to undesirable biases in the latent space. For instance, gender, race, or cultural biases can manifest in specific dimensions, causing the model to make biased predictions or generate unfair outputs. Analyzing the latent space for biased subspaces or skewed distributions is crucial for addressing fairness issues in NLP models. Techniques such as adversarial debiasing or fair representation learning can help mitigate these biases.

    \item \textbf{Dimensional Collapse}: Dimensional collapse occurs when the embeddings occupy only a small subset of the available dimensions, effectively reducing the dimensionality of the learned representations. This can happen due to overfitting or excessive regularization, leading to embeddings that do not utilize the full capacity of the vector space. Analyzing the effective dimensionality and ensuring that embeddings utilize the available space can help in mitigating this issue.

    \item \textbf{Loss of Geometric Structure}: Effective embeddings should maintain meaningful geometric relationships, such as clustering of similar words and separation of dissimilar ones. Degeneration can lead to a loss of these geometric structures, where embeddings fail to reflect semantic or syntactic relationships accurately. Techniques like manifold learning or embedding space visualization can be used to analyze and restore geometric properties.

    \item \textbf{Regularization and Hyperparameter Tuning}: Overly aggressive regularization can cause embeddings to collapse into subspaces, while insufficient regularization might lead to overfitting. Proper tuning of regularization parameters and hyperparameters is crucial for maintaining the balance between effective representation learning and avoiding degeneration.

\end{itemize}

\textbf{Mitigating Geometric Degeneration} involves several strategies:
\begin{itemize}
    \item \textbf{Enhanced Training Techniques}: Using more diverse and extensive datasets can help the model learn richer representations and prevent degeneration. Techniques such as data augmentation and noise injection can also improve the robustness of embeddings.
    
    \item \textbf{Dimensionality Analysis}: Analyzing and managing the dimensionality of the embedding space helps in ensuring that the model makes full use of the available dimensions. Methods like Principal Component Analysis (PCA) and Singular Value Decomposition (SVD) can help in identifying and addressing dimensional collapse.

    \item \textbf{Bias Mitigation Techniques}: Applying techniques like adversarial training, counterfactual data augmentation, and fairness constraints can help in reducing biases in the latent space and ensuring more equitable representations.

    \item \textbf{Regularization Techniques}: Applying appropriate regularization techniques to prevent overfitting and ensure embeddings retain their discriminative power. Techniques like weight decay, dropout, and layer normalization should be carefully tuned.

    \item \textbf{Model Architecture Adjustments}: Modifying the model architecture to increase capacity or adjust attention mechanisms can help in learning better representations and addressing geometric issues. For example, increasing the number of attention heads or layers can enhance the model’s ability to capture complex relationships.
\end{itemize}

In summary, representation degeneration in the geometric sense involves anisotropy, outlier dimensions, representation collapse, and biases in latent spaces. Addressing these issues is crucial for maintaining the effectiveness of learned embeddings and ensuring that they provide accurate and fair representations in NLP models.


\section{Beyond Classical Language Modeling}

This section explores advancements and innovations that extend beyond traditional language modeling approaches. These innovations aim to improve various aspects of model performance, efficiency, and flexibility.

\subsection{Tokenizer-Free Language Modeling}

Traditional language models rely heavily on tokenization to convert text into discrete units that the model processes. However, recent approaches are moving towards tokenizer-free language modeling, which seeks to bypass or minimize the need for tokenization.

\begin{itemize}
    \item \textbf{Direct Subword Encoding}: Some models directly operate on raw text by learning representations at the character or subword level, eliminating the need for predefined tokenization schemes. This can improve handling of rare or out-of-vocabulary words and reduce preprocessing complexity.
    
    \item \textbf{Byte-Level Models}: Byte-level language models work directly with raw byte sequences, which allows them to handle any character set and avoid the limitations of fixed vocabulary sizes. These models learn to process text without the need for explicit tokenization, enabling more flexible and universal text representation.

    \item \textbf{End-to-End Models}: End-to-end models process text directly from input to output, bypassing intermediate tokenization steps. These models can be more adaptable and efficient in certain applications, such as real-time text generation or language understanding tasks where tokenization might introduce latency.

\end{itemize}

\subsection{Efficient Attention}

Efficient attention mechanisms aim to address the computational and memory inefficiencies associated with traditional self-attention mechanisms, especially in large-scale models.

\begin{itemize}
    \item \textbf{Sparse Attention}: Sparse attention mechanisms focus on attending to a subset of tokens rather than the entire sequence. Techniques like local attention, where only nearby tokens are attended to, and global attention, where certain tokens receive full attention, reduce the complexity and improve efficiency.

    \item \textbf{Low-Rank Approximations}: These methods approximate the attention matrix with low-rank representations, reducing computational and memory requirements. Approximations like LinFormer and Performer utilize mathematical techniques to simplify the attention computation.

    \item \textbf{Memory-Augmented Attention}: Memory-augmented attention mechanisms introduce external memory structures to store and retrieve information, reducing the need to compute attention over the entire sequence. This can be particularly useful for long sequences or tasks requiring long-term dependencies.

    \item \textbf{Kernel-Based Attention}: Kernel-based attention techniques approximate the attention mechanism using kernel functions, which can significantly speed up computations. Examples include the Reformer and LinFormer models, which utilize kernel-based methods to achieve linear time complexity.

\end{itemize}

\subsection{Alternative Training Tasks \& Objectives}

Alternative training tasks and objectives explore new ways to train language models beyond traditional language modeling objectives, such as predicting the next token.

\begin{itemize}
    \item \textbf{Contrastive Learning}: Contrastive learning techniques train models by contrasting positive and negative examples, improving the quality of learned representations. Approaches like SimCLR or MoCo adapt this framework to NLP by learning embeddings that are close for similar examples and far apart for dissimilar ones.

    \item \textbf{Multi-Task Learning}: Multi-task learning involves training a model on multiple tasks simultaneously, enabling it to generalize better across different domains. By sharing representations across tasks, models can leverage complementary information and improve performance on each individual task.

    \item \textbf{Masked Language Modeling (MLM)}: MLM involves masking portions of the input text and training the model to predict the masked tokens. This approach, used in models like BERT, helps the model learn bidirectional context and capture more nuanced language understanding.

    \item \textbf{Denoising Autoencoders}: Denoising autoencoders are trained to reconstruct corrupted input text. This training objective helps the model learn robust representations by focusing on recovering clean text from noisy or incomplete inputs, which improves generalization to various tasks.

    \item \textbf{Self-Supervised Objectives}: Self-supervised learning tasks generate supervisory signals from the data itself, reducing the need for labeled examples. Tasks like next-sentence prediction, sentence permutation, and sentence similarity help models learn useful features from raw text data.

\end{itemize}

In summary, the advancements beyond classical language modeling include tokenizer-free approaches that simplify text processing, efficient attention mechanisms that address computational challenges, and alternative training tasks and objectives that enhance model learning and performance. These innovations contribute to more effective and adaptable language models, pushing the boundaries of what is achievable with NLP technologies.



